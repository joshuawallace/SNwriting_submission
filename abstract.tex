We calculate the distance-dependent performance of a few representative terrestrial 
neutrino detectors in detecting and measuring the properties of the $\nu_e$
breakout 
burst light curve in a Galactic core-collapse supernova. The breakout 
burst is a signature phenomenon of core collapse and offers a probe
into 
the stellar core through collapse and bounce. We examine cases of 
no neutrino oscillations and oscillations due to normal and inverted 
neutrino-mass hierarchies. For the normal hierarchy, other neutrino 
flavors emitted by the supernova overwhelm the $\nu_e$ signal, making a 
detection of the breakout burst difficult. For the inverted hierarchy, 
some detectors at some distances should be able to see the $\nu_e$ breakout 
burst peak and measure its properties. For the inverted hierarchy, the 
maximum luminosity of the breakout burst can be measured at 10 kpc to 
accuracies of \abt 30\% for Hyper-K and \abt 60\% for DUNE. Super-K and JUNO lack 
the mass needed to make an accurate measurement. 
Detector backgrounds in IceCube render a measurement of the $\nu_e$
breakout burst unlikely. For the inverted
hierarchy, 
the time of the maximum luminosity of the breakout burst can be
measured 
in Hyper-K to an accuracy of \abt 3 ms at 7 kpc, in DUNE \abt 2 ms at 4 kpc, 
and JUNO and Super-K can measure the time of maximum luminosity to 
an accuracy of \abt 2 ms at 1 kpc. For the inverted hierarchy, a 
measurement of the maximum luminosity of the breakout burst could 
be used to differentiate between nuclear equations of state. 
