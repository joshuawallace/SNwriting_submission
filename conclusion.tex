We have calculated the expected performance of several representative
terrestrial neutrino detectors in detecting and measuring the
properties of the \nue\ breakout burst light curve in the event of a Galactic
CCSN as a function of supernova distance. We have also examined
whether these measurements of the breakout burst peak would be
sufficiently accurate to allow discrimination between different CCSN
progenitor models and nuclear EOSs.  We have explored the case
of no neutrino oscillations and neutrino oscillations due to
both the normal and inverted neutrino-mass hierarchies.  Assuming
Gd doping in water-\cer\ detectors, in the no-oscillation case
 backgrounds to the \nue\ signal 
due to other neutrino flavors emitted by the CCSN
are sufficiently low as to be negligible, allowing for the best
detection of the \nue\ breakout burst peak and best measurement of its
properties. Neutrino oscillations serve to both reduce the
detectability of the original \nue\ flux and increase the
detection rate of the \background.  We show that, in the NH
case, the \backgrounds\ are too large relative to the detectable original
\nue\ flux to see the \nue\ breakout burst peak for any of the
detectors under consideration in this work.  In the IH case,
three of
the detector types examined (water-\cer, scintillation, and
\ar40) would have a distance-dependent ability to see the
\nue\ breakout burst peak and measure its properties, although to less
accuracy than in the no-oscillation case.  
A long-string detector like IceCube, even in the IH case, would be unable by itself to
    detect the breakout burst peak.  Additionally, the random fluctuations in
    IceCube's background rate would swamp any signal that could be extracted about the 
    \nue\ light curve for SNe at reasonable SN distances even in 
    the no-oscillation case.


The maximum luminosity of the breakout burst, \lmax, can be measured
to the following errors for a CCSN at 10 kpc in the case of no
oscillations:  \abt 25\% for Hyper-K and \abt 30\% for DUNE.  
Super-K and JUNO have very large errors ($>$100\%) at
10 kpc, but JUNO would be able to make a measurement to \abt 40\%
error and Super-K would be able to make a measurement to \abt 60\%
for an SN at 4 kpc.  
In the oscillation case due to the IH, \lmax\ can be measured to the
following errors for a CCSN at 10 kpc: \abt 30\% for Hyper-K and
\abt 60\% for DUNE.  Super-K and JUNO again have very
large errors at 10 kpc. At 1 kpc, JUNO would have an error of \abt 15\% 
and Super-K would have an error of \abt 20\%.
A \abt 2\% accuracy would be needed to
differentiate between the progenitor masses examined in this work (12,
15, 20, and 25 \Msol).  In the no-oscillation case, 
Hyper-K is close to attaining this accuracy 
for SNe out to \abt 1 kpc, but no other
detector could do so for distances ${\geq}1$ kpc.  In the
IH-oscillation case, no detector in this study could make a
sufficiently accurate determination of \lmax\ for any distances examined. 
A 10\% accuracy is
needed in a determination of \lmax\ 
to differentiate between the \ls\ and the \shen\ for the
15-\Msol\ progenitor.  In the no-oscillation case, 
this accuracy is attained by Hyper-K and DUNE out to \abt 4 kpc, 
 and by JUNO out to \abt 1 kpc.  Super-K is close to achieving this
 accuracy at 1 kpc but does not achieve this accuracy for any of the
 distances examined in this work.
In the IH-oscillation case, this accuracy is attained by DUNE out to
\abt 2 kpc and by Hyper-K out to \abt 2-3 kpc. 

The time of the maximum luminosity of the breakout burst, \tmax, can
be measured to the following accuracies for a CCSN at 10 kpc in the
case of no oscillations: \abt 1.3 ms for Hyper-K and \abt 1.5 ms for DUNE.
JUNO has an error of \abt 2 ms at 4 kpc, and Super-K has an error
of \abt 2.5 ms at the same distance.
In the case of IH oscillations, Hyper-K could measure \tmax\ with an
accuracy of \abt 3 ms at 7 kpc, DUNE could measure \tmax\ with an
accuracy of \abt 2 ms at a distance of 4 kpc, JUNO 
could measure \tmax\ to an accuracy of \abt 1.5 ms at 1 kpc, and
Super-K could achieve an accuracy of \abt 2 ms for the same distance.
\tmax\ is
not useful in discriminating between CCSN models and EOSs, 
but may be useful in triangulating the position of an 
SN. Back-of-the-envelope estimates performed by us predict that
a triangulation incorporating \tmax\ information from either Super-K
or Hyper-K would not provide a more accurate determination of the
location of the SN than the individual pointing information
available in these water \cer\ detectors.  However, an optimal
alignment relative to the baselines between detectors may provide a
triangulation measurement of comparable (though lesser) accuracy to an
individual detector pointing, and both location measurements could be
productively combined.  We reserve a more complete examination of the
triangulation abilities of measurements of \tmax\ for a future study.


The width of the breakout burst peak, $w$, can be measured to the
following accuracies for a CCSN at 10 kpc in the case of no
oscillations: \abt 4 ms for Hyper-K and \abt
5 ms for DUNE.  JUNO and Super-K do not observe sufficient numbers of
neutrinos for SNe at 10 kpc to make accurate determinations of
$w$, but at 1 kpc JUNO could make a measurement of $w$ to \abt 1.5 ms
accuracy and Super-K could achieve an accuracy of \abt 2 ms, in the case of no oscillations.  One is unable to measure
$w$ in the IH case because the rising \backgrounds\ make the fall from
peak \nue\ less clear than in the no-oscillation case and measuring
the width is difficult.

Measurements of \trise\ and \tfall\ could be used
to show that \tfall$>$\trise.    In the
no-oscillation case, 
Super-K would be able to confirm  \tfall$>$\trise\ out to \abt 2 kpc, JUNO would be able to out to \abt 3 kpc, 
DUNE would be able to out to \abt 10 kpc, and Hyper-K would be able to
out to \abt 11-12 kpc.  A
determination of \tfall$>$\trise\ is difficult to make in the NH and
IH cases because \tfall\ is difficult to measure owing to  
the increasing dominance of the \backgrounds\ at
the time of fall from the \nue\ peak in these cases.



If the \backgrounds\ could be removed while leaving the \nue\ signal intact, 
the results presented in this work with oscillations due to the NH and
IH would be improved.
 The loss of the detectability of the
original \nue\ flux after it oscillates (fully or partially) to
\nuxpart\ flux cannot be made up this way, but a statistical subtraction
of the backgrounds from other neutrino species could allow for a
measurement of the \nue\ breakout burst peak in the NH case and could improve peak detectability
and measurements in both cases.  In principle, a complete statistical
subtraction of all the \backgrounds\ is possible.
For the IH, the \anue\ flux is completely exchanged with one of 
the \nuxanti's.  For Gd-doped water-\cer\ or scintillation detectors,
which are particularly sensitive to \anue's,
this would allow for a good measurement of the original \nuxanti\ flux, 
which in turn can be translated to the \nuxpart\ flux well enough for
these to all be statistically subtracted.  This would leave only the
original \nue\ and \anue\ flux in the signal.
 However, since in the IH there is still some original \nue\ flux 
that remains intact, the signal in these detectors (especially through
the \nue\ peak, which occurs before the original \anue\ flux begins to
rise) will be dominated by \nue's.

A similar subtraction of the \backgrounds\ in the NH is a little more
complicated, but still possible. However, it would 
require data from multiple
detectors. The original \anue\ flux only partially oscillates to
\nuxanti, making a complete subtraction of the \nuxpart\ and \nuxanti\ flux
using the principle described above less straightforward.  
In this case, an \ar40\ detector will probably be the most useful to
subtract the \backgrounds.  The \nue's it measures are originally
\nuxpart's. Although one might not be able to disentangle the signals
from the various detection channels in the detector,  
the \nue\ signal (from what was originally the \nuxpart\ flux) 
will be the dominant signal.  Thus, an
\ar40\ detector should be able to measure the original \nuxpart\ flux and, 
using this measurement, should be able to statistically subtract 
all the \nuxpart\ and \nuxanti\ flux.  The 
measurement of \anue's in proton-rich detectors, with the 
\nuxanti-oscillated-to-\anue\ flux subtracted off, will provide a 
measurement of \anue's, which can then be statistically 
subtracted off as well, leaving behind only the \nue\ flux.  Both the
IH and NH cases can also benefit from a measurement of the \nux\ flux
from scintillation detectors (\citealt{lahabeacom2014}).

The success of these procedures in subtracting the \backgrounds\ 
depends on distance, since a larger distance means a lower flux and
a less precise measurement of the \background, which precision would
propagate to the extracted \nue\ signal.  We encourage the various
collaborations associated with the extant and future neutrino detectors
to examine this topic and continue to investigate methods to identify
and subtract the \backgrounds.


The improvements that have been made in neutrino detection
technologies since SN 1987A have put the scientific
community in a good position to take full advantage of the neutrino
emission from the next Galactic CCSN.  In particular, the
\nue\ breakout burst peak (if it exists) from a Galactic CCSN 
could be detectable (depending on distance) 
in current and near-future neutrino detectors in
the case of the IH, but it likely won't be detectable in the NH case
(although sufficient \background\ subtraction could allow the \nue\
peak to be detected).  
A detection or nondetection of the \nue\ breakout
burst peak by itself should be sufficient to identify the
neutrino-mass 
hierarchy ({\citealt{mirizzietal2015}), and a measurement of the
  properties of the breakout burst could constrain 
progenitor mass and the nuclear EOS.  The rapidly
maturing fields of neutrino physics and neutrino astrophysics will be
greatly served by the next Galactic CCSN.

