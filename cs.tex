This Appendix lists the sources used for the neutrino
interaction cross sections relevant to our calculations. 
If the analytic cross section is known, then it is presented here.
Otherwise, reference is made to the source of the tabulated values of
the cross section.  
Figure~\ref{fig:sigma} shows the neutrino-matter-interaction
cross sections for \nue\ and \anue.
In what follows, it is useful to
define the quantity $\sigma_0$ as
\begin{equation}
\sigma_0 = \frac{4G_\textrm{F}^2\cos^2{\theta_c}(m_ec^2)^2}{\pi(\hbar c)^4}
\simeq 1.705 \times 10^{-44}~\textrm{cm}^2,
\end{equation}
where $G_\textrm{F}$ the Fermi constant, 
 $\theta_c$ is the Cabbibo angle, and $m_e$ is the electron mass. 




Elastic scattering off of electrons ($\nu_i + e^- \rightarrow \nu_i +
e^-$)  is the primary $\nu_e$ detection channel for all but
\ar40\ detectors. \cite{tomasetal2003} provide the
differential cross section, given by
%
\beq   \label{el}
 \frac{d\sigma}{dy} = 
 \frac{\sigma_0}{8\cos^2{\theta_c}}\frac{E_\nu}{m_ec^2} \left[ A+B\,(1-y)^2-C\,\frac{m_e}{E_\nu}\,y
 \right] ~,
\eeq         
%
where $y=E_e/E_\nu$ is the energy fraction transferred to the electron.  The
coefficients $A$, $B$ and $C$ differ for the four different reaction
channels and are given in Table~\ref{ABC} (based on a similar table in
\citealt{tomasetal2003}). The vector and axial-vector
coupling constants have the usual values $C_V
=-\frac{1}{2}+2\sin^2\Theta_W$ and $C_A=-\frac{1}{2}$, with
$\sin^2\Theta_W \approx 0.231$ (\citealt{oliveetal2014}) being the Weinberg angle.

In an electron scattering, the relationship between the energy fraction transferred to the
electron ($y$) and the scattering angle
$\theta$ is given by (\citealt{tomasetal2003})
%
\beq
 y = \frac{2\,(m_ec^2/E_\nu)\,\cos^2\theta}{(1+m_ec^2/E_\nu)^2-\cos^2\theta}
 ~. 
\eeq
%
The total cross sections for electron scattering are given by
\cite{marcianoparsa2003}.
 Ignoring corrections of order $m_ec^2/E_\nu$,  the following are the 
  total cross sections:
\begin{align}
\sigma (\nu_e + e^- \rightarrow \nu_e + e^-) &= \frac{\sigma_0}{8\cos^2{\theta_c}}\left(\frac{E_\nu}{m_ec^2}\right) [1+4\sin^2\theta_W + \frac{16}{3} \sin^4\theta_W],
\end{align}
\begin{align}
\sigma (\bar\nu_e + e^- \rightarrow \bar\nu_e + e^-)  &= \frac{\sigma_0}{8\cos^2{\theta_c}}\left(\frac{E_\nu}{m_ec^2}\right)[\frac13+\frac43\sin^2\theta_W + \frac{16}{3} \sin^4\theta_W].
\end{align}
The \nux\ total cross sections are given by (neglecting terms of
 order $m_ec^2/E_\nu$)
\begin{align}
\sigma (\nu_{\mu,\tau} + e^- \rightarrow \nu_{\mu,\tau} + e^-)  &= \frac{\sigma_0}{8\cos^2{\theta_c}}\left(\frac{E_\nu}{m_ec^2}\right) [1-4\sin^2\theta_W + \frac{16}{3} \sin^4\theta_W],
\end{align}
\begin{align}
\sigma (\bar\nu_{\mu,\tau} + e^- \rightarrow \bar\nu_{\mu,\tau} + e^-) &= \frac{\sigma_0}{8\cos^2{\theta_c}}\left(\frac{E_\nu}{m_ec^2}\right)[\frac13-\frac43\sin^2\theta_W + \frac{16}{3} \sin^4\theta_W].
\end{align}
For IBD ($\bar\nu_e + p \rightarrow n + e^-$)
we use the analytic cross section of \cite{burrowsetal2006}, given by
\begin{equation}
\label{eq:anuep}
\sigma({\bar\nu_e p \rightarrow N + e^-}) =
\sigma_0\frac{1+3g_A^2}{4}\left(\frac{E_{\bar\nu_e}-\Delta_{np}}{m_ec^2}\right)^2
\left[1-\left(\frac{m_ec^2}{E_{\bar\nu_e} - \Delta_{np}}\right)^2\right]^{1/2} W_{\overline{M}},
\end{equation}
where $g_A$ is the axial-vector coupling constant, $\Delta_{np}$ is the mass-energy difference between a
proton and a neutron $(m_n - m_p)c^2$, and
$W_{\overline{M}}$ 
is the correction for weak magnetism and
recoil, $(1-7.1 E_{\bar\nu_e}/m_nc^2)$.
For both \nue\ and \anue\ CC absorption on \ar40, 
we use the cross sections from \cite{kolbeetal2003}, their Figure 9.
The data for the oxygen cross sections (which include CC
absorption by \nue\ and \anue\ and NC scattering by all species) are taken from tables in
\cite{kolbeetal2002}.  In our
analysis, we assume that all oxygen is $^{16}$O.
The cross sections for CC absorption of \nue\ and \anue\ on carbon are taken from tables in
\cite{kolbeetal1999}.  The cross sections for NC
scattering of all neutrino flavors on carbon are taken from tables in  \cite{fukugitaetal1988}.
In our analysis, we assume that all carbon is $^{12}$C.  
Neutrino
elastic scattering off of protons ($\nu_i + p \rightarrow \nu_i + p$)
is also expected to be detectable in scintillation detectors owing to
their low detection thresholds (\citealt{beacom2002}).  
However, the flux primarily probed by
this channel will be the \nux\ flux (referring to the what is
the \nux\ flux before oscillations occur) because of its higher average
energy.  \cite{oberaueretal2005} state that the 
signal from the recoil protons can be easily separated from the other
signals.  We assume this ability in our analysis and do not include
contributions from NC scattering on protons.
\begin{deluxetable}{lccc}
\tablecolumns{4}
\tablecaption{\label{ABC} 
Coefficients used in Eq.~(\ref{el}) for the elastic scattering of
neutrinos on electrons.}
\tablehead{ Neutrino Type & \colhead{$A$} & \colhead{$B$} & \colhead{$C$} }
%   & $A$ & $B$ & $C$ \\[0.5ex] \hline  
\startdata
$\nu_e$ & $(C_V{+}C_A{+}2)^2$ & $(C_V-C_A)^2$ & $(C_V+1)^2-(C_A+1)^2$ \\
$\bar\nu_e$ & $(C_V-C_A)^2$ & $(C_V{+}C_A{+}2)^2$ & $(C_V+1)^2-(C_A+1)^2$\\
$\nu_{\mu,\tau}$ & $(C_V+C_A)^2$ & $(C_V-C_A)^2$ & $C_V^2-C_A^2$\\
$\bar\nu_{\mu,\tau}$ & $(C_V-C_A)^2$ & $(C_V+C_A)^2$ &
$C_V^2-C_A^2$ 
\enddata
\end{deluxetable}




