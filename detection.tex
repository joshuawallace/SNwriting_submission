 There are four main classes
of detectors relevant for detecting the \nue\ 
breakout burst: \ar40\ detectors, water-\cer\
detectors, long-string detectors, and 
scintillation detectors.    
Each of these detector classes has its
own strengths and weaknesses with regard to the detection of
the breakout burst.

Table~\ref{tab:detectors} lists the detectors we highlight in our
study, as well as some other detectors of interest for detection of
the breakout burst.  
It consists both of detectors currently running and detectors
that are expected to come online in the coming years.  The list is not
exhaustive, but is rather a representative ``short-list'' of detectors we
think will provide the best opportunity to examine the breakout
burst.  We now discuss specifics of each class of detector.

\begin{deluxetable*}{lccc}
%\begin{deluxetable}{cccc}
\tablecolumns{4}
\tablecaption{Selected Neutrino Detectors\label{tab:detectors}}

\tablehead{\colhead{Detector} & \colhead{Detection Medium} & \colhead{Mass} & \colhead{Status} \\ 
\colhead{} & \colhead{} & \colhead{(ktonne)} & \colhead{} } 

\startdata
DUNE &  \ar40 & 40 & planning\tablenotemark{a}\\
Hyper-K & water & 560 (fiducial) & proposed\tablenotemark{b}\\
Super-K & water & 22.5 (fiducial)\tablenotemark{c} & running \\
IceCube & water ice & $\sim$900\tablenotemark{d} & running\\
JUNO  & scintillator & 20\tablenotemark{e} & construction\\
\hline\\ [-1.25ex]

ICARUS & \ar40 & .476 (active)\tablenotemark{f} & being refurbished\\
KamLAND & scintillator & 1\tablenotemark{g}  & running  \\ 
LVD & scintillator & $\sim$1\tablenotemark{h} & running \\
NO$\nu$A\tablenotemark{i} & scintillator & 14\tablenotemark{j} &
running
\enddata

\tablenotetext{a}{Mass and status from \cite{deoliviera2015};}
\tablenotetext{b}{Mass and status from \cite{abeetal2011};}
\tablenotetext{c}{Mass from \cite{ikedaetal2007};}
\tablenotetext{d}{Based on the energy-dependent effective detection volume given in \cite{abbasietal2011};}
\tablenotetext{e}{Mass from \cite{li2014};}
\tablenotetext{f}{Mass from \cite{rubbiaetal2011};}
\tablenotetext{g}{Mass from \cite{eguchietal2003};}
\tablenotetext{h}{Mass from http://www.bo.infn.it/lvd/;}
\tablenotetext{i}{NO$\nu$A is located at the surface;}
\tablenotetext{j}{Mass from \cite{pattersonetal2012}.}
\end{deluxetable*}
%\end{deluxetable}


\subsection{\ar40\ Detectors}
Of all the detector types we consider, 
\ar40\ detectors have the highest sensitivity to \nue's (the primary
neutrino emission of the breakout burst, ignoring oscillations). The 
\nue's interact
with \ar40\ nuclei via charged-current (CC) 
capture; it is the large cross section of
this interaction  that gives \ar40\ 
detectors such great \nue\ sensitivity.  The electron created in this
process deposits its kinetic energy 
along an ionization trail through the detection medium.  The \ar40\
detectors we mention  in this paper are all time-projection chambers
(TPCs).  In a TPC, a voltage is applied
across the detection medium, causing the particles ionized by the
product electrons to drift
toward a wire mesh that (when combined with the timing of the
formation of the ionization trail) gives spatial information about the
interaction inside the detector.  The \anue's also undergo CC
absorption with the \ar40\ nuclei.  The cross section for this
interaction, for the neutrino energies in a supernova, is 2-3
orders of magnitude smaller than for \nue\ CC absorption
in the energy range relevant for this study,
and so this interaction serves as a small background to the \nue\
signal in the case of no oscillations.  Neutrinos of all flavors can also be
detected  via elastic scattering off the electrons.
Electrons produced from electron scattering are 
indistinguishable from the electrons produced by CC
absorption of \nue's on the \ar40\ nuclei, but electron scattering has a
much smaller cross section (factor of ${\sim}100$ for 10 MeV neutrinos) 
than the CC interaction channel on \ar40.
 Because of the dominance of
the cross sections through which \nue's are detected and the dominance
of the \nue\ flux relative to the \anue\ and \nux\ flux through the breakout
burst, we assume the ability of these detectors to separate or ignore
the background \anue\ and \nux\ signals 
and consider only the \nue\ luminosity in our analysis for the case of
no neutrino oscillations.  The mixing of the \nuxpart\ and \nue\ flux
in the case of neutrino oscillations complicates this and we cannot
assume, for either the NH or IH, that the \nuxpart\ backgrounds are negligible. 
Also, \ar40\ detectors can measure
the energies of the electrons produced in the neutrino interactions
inside them.  
In principle, gamma-rays from nuclear de-excitation
could be detected, allowing the tagging of \nue\ CC
absorption events and their separation from electron scattering
events, as well as detection of neutral-current (NC) scatterings off of
nuclei.  The detectability of these gamma-rays is still under study
(A. Rubbia 2015, private communication),
and we do not assume the ability to detect them.

The largest \ar40\ detector operated so far is ICARUS (Imaging 
Cosmic And Rare Underground Signals), formerly located in the Gran Sasso 
underground laboratory in Italy.  The detector had an active mass of
476 tonnes (\citealt{rubbiaetal2011}).  The detector is currently in
the process of being refurbished for later installation in the USA 
at Fermilab.  
In our analysis, we assume a detection efficiency
of 100\% across all product energies for the \ar40\ detectors. 
We note that the interaction cross sections set their own
threshold for neutrino detection, which is incorporated with our
cross-section 
calculations.  We refer interested readers to the Appendix for
a further discussion of these cross sections.  
We calculate that a detector of ICARUS's size 
will detect $\sim$1 \nue\ in a 10-ms period over the
breakout burst at a distance of 10 kpc in the case of no neutrino
oscillations (even less in either NH or IH case oscillations). 
Because of the expected small
signal from ICARUS, we do not use it in our analysis.

There are plans for a 40-ktonne (fiducial) \ar40\ detector to be
constructed at the Sanford Lab in the
Homestake Mine in South Dakota as a part of
part of the Deep Underground Neutrino Experiment    
 (DUNE; formerly the 
Long Baseline Neutrino Experiment, LBNE; 
\citealt{deoliviera2015}).  We calculate that an \ar40\ detector of
this size will detect $\sim$120 \nue's in a 10-ms period over the
breakout burst at a distance of 10 kpc in the case of no neutrino
oscillations.  For the same situation as just outlined, the number of
detected (original) \nue's in the NH case will be \abt 2 and in the IH
case will be \abt 40.  The current plans for DUNE
may leave it with photon timing only as precise as \abt 1-3 ms.  We
assume an upgraded DUNE that will be able to measure photon timing to
much better than the time bin width used in our analysis (1 ms).



\subsection{Water-\cer\ detectors}
Water-\cer\ detectors are large tanks of purified water
primarily sensitive to \anue's through inverse $\beta$ decay (IBD) on
protons (hydrogen nuclei): $\bar\nu_e + p \rightarrow e^+ + n$. 
The positron produced by IBD emits \cer\ light, which is detected by
photomultiplier tubes placed around the detection
volume. Neutrinos of all flavors can be detected through elastic
scattering on electrons, $\nu_i + e^- \rightarrow \nu_i + e^-$.  
The final state electrons are detected through their
\cer\ emission.  \nue\ and \anue\ can also undergo CC absorption 
on the oxygen
nuclei  and are detected through the electrons/positrons formed in
these interactions, as well as through photons emitted via nuclear
deexcitation.  
Additionally, neutrinos of all types may undergo NC
interactions with the oxygen nuclei, which (if the interaction puts
the nucleus in an excited state) can be detected by the photon emitted
upon deexcitation of the nucleus.

In standard water-\cer\ detectors, positrons from
IBD's and electrons from 
electron scatterings can be statistically distinguished by the
forward-peaked directionality of the electron scattering products and the
nearly-isotropic products of IBD.
 The IBD
detection channel has the largest cross section and so 
dominates when there is a \anue\ flux.  Figure~\ref{fig:lumallt} shows
that the \anue\ flux is not significant until after the
\nue\ peak of the breakout burst.
Thus, previous to this almost all detections can be attributed
to \nue's (in the no-oscillation case).
After this the \anue's (and, less so, the \nux's) 
must be accounted
for.  Future water-\cer\ detectors may
employ gadolinium (Gd) to tag the final-state neutrons and allow the IBD and
electron elastic scattering signals to be separated
(\citealt{vagins2012}; \citealt{lahabeacom2014}).  The large neutron
capture cross section of Gd allows neutrons formed in IBD events to be
quickly ($\sim$20 $\mu$s) captured, emitting 3-4 gamma-rays with a total
energy of 8 MeV (\citealt{beacomvagins2004}).  
The coincidence of a neutrino
detection and a gamma ray from neutron capture on Gd 
allows the neutrino detection to be associated with an IBD.  We
assume the presence of Gd in our analysis.  In practice, using Gd to
tag IBD events will not be perfect, although the fraction of neutrons
that are captured onto Gd is very high even for modest additions of Gd
to water (\citealt{beacomvagins2004}).  The signal remaining from
untagged IBD's can be statistically subtracted from the remaining signal.
Because of this, and because of the ability to statistically distinguish
\nue's and \anue's based on direction, in our analysis we assume that
IBD's can be separated from the rest of the signal.  In the case of no
oscillations, this leaves the \nue\ flux dominant.
\cite{lahabeacom2014} also state that \nux\ 
information from scintillation detectors 
will allow those events to be statistically subtracted,
leaving only the \nue\ events.  Based on this, we assume the
ability to subtract or ignore all \backgrounds\footnote{In this work,
the \backgrounds\ to which we refer are due to \anue\ and \nux\
emission from the CCSN itself, not the detector backgrounds or any
ambient neutrino background (for instance, solar neutrinos or the
diffuse supernova neutrino background).} in the no-oscillation
case.  In the case of neutrino oscillations, the detectability of the 
background \nux\ flux
will increase and cannot be so easily ignored.

Super-Kamiokande (Super-K) is a 22.5-ktonne (fiducial) water-\cer\ 
detector located
in the Kamioka Mine in Japan (\citealp{fukudaetal2003,ikedaetal2007}).
Super-K
 IV, for solar
 neutrino analysis, reports a 99\% triggering efficiency for 4.0-4.5 
MeV and a 100\% triggering efficiency above 4.5 MeV
(\citealt{sekiya2013}).  The exact threshold and detection volume that
can be used for neutrino detection in a CCSN depend on the background
rate and the signal rate, the latter of which depends on distance to
the supernova.  Thus, there is some ambiguity regarding which detector
threshold would be most appropriate to use.
In our analysis, we assume a Heaviside step function with 
step at 4.0 MeV as our detection efficiency function for Super-K, for
all supernova distances.  Since electron scattering
has a spread of electron energies that can be produced for a given
neutrino energy, neutrinos with energies well above the detector
threshold may result in product electrons that fall below the
threshold.  For 10 MeV \nue's (comparable to the average neutrino
energy through the breakout burst), we find that \abt 40\% of electron
scatterings produce electrons with energies below our chosen 4 MeV
threshold.
This represents a significant reduction in the detected \nue\ flux
relative to what otherwise could be measured that could possibly be measured. 
Any improvements that can be made to the detector threshold for
a CCSN has significant potential to 
improve the results presented throughout this work.
  We calculate 
that Super-K  will detect $\sim$7 \nue's in a 10-ms period over the
breakout burst at a distance of 10 kpc in the case of no neutrino
oscillations.  For the same situation as just outlined, the number of
detected (original) \nue's in the NH case will be \abt 0 and in the IH
case will be \abt 1.

Super-K has recently been approved to have Gd added, following
years of study as to the  feasibility and impact of adding Gd to 
Super-K (\citealp{beacomvagins2004,watanabe2009,vagins2012,mori2013}).  
\cite{beacomvagins2004} suggest that with 0.2\% (by mass) 
Gd added to Super-K,
${\sim}90\%$ of the IBD events could be tagged. The remaining
IBD events (as well as the \anue\ absorption events on $^{16}$O) 
can then be statistically subtracted from the remaining
signal.  

Hyper-Kamiokande (Hyper-K) is a proposed 560-ktonne (fiducial) water-\cer\
detector planned to be located at the Kamioka Mine in Japan
(\citealt{abeetal2011}). The detector threshold for Hyper-K depends on
the photomultiplier tube coverage fraction, which is not yet
finalized.  If this coverage fraction is smaller than that of Super-K,
then the detector threshold for Hyper-K will be greater than for
Super-K.  Because the final detector design is not yet finalized and
for simplicity, for Hyper-K, we assume the same 4 MeV 
detector threshold
we assume for Super-K.  We calculate 
that a water-\cer\ detector of Hyper-K's size  will detect 
$\sim$160 \nue's in a 10-ms period over the
breakout burst at a distance of 10 kpc in the case of no neutrino
oscillations.  For the same situation as just outlined, the number of
detected (original) \nue's in the NH case will be \abt 30 and in the IH
case will be \abt 70.

Both Super-K and Hyper-K could provide an estimate of the direction of the neutrino 
flux.  The scattering of \nue's  off of electrons results in electron 
propagation that is forward peaked relative 
to the incident neutrino's motion.  The direction of
the final-state electrons can be measured using information from the
electrons' \cer\ light cones.  Because the electrons produced in these
scatterings are not perfectly forward-peaked, and because of the
subsequent straggling of the electrons as they scatter within the
detector, the precision of such a direction measurement is
limited.  \cite{andosato2002} calculate that Super-K can measure the
location of a CCSN at 10 kpc to within a circle of
$\sim$9\degree\ radius, using \nue's measured over the whole neutrino
event (and not just the breakout burst).  \cite{tomasetal2003}
calculate that Super-K, with Gd added, could measure a supernova
position to an accuracy of 3.2\degree-3.6\degree,
depending on the neutron tagging efficiency.  They also calculate that
a megatonne water detector with Gd and 90\% tagging efficiency would measure
the direction to an accuracy of 0.6\degree, and  \cite{abeetal2011} state
that Hyper-K would be able to measure direction for a CCSN at 10 kpc
to an accuracy of $\sim$2\degree.

\subsection{Long-String Detectors}

IceCube is a long-string detector embedded in the Antarctic ice at the
South Pole (\citealp{achterbergetal2006,abbasi2010}). It
is optimized for the detection of neutrinos with TeV energies, much
higher than the $\mathcal{O}(10)$~MeV energies expected for the
breakout-burst neutrinos. 
However, the
neutrinos from a CCSN will create a correlated rise in the measured
detector background across all the individual detection components of IceCube
(\citealp{pryoretal1988,halzenetal1996}).  
%Estimates for the 
%volume probed for CCSN neutrino
%detection by the entire
%ensemble of detectors in IceCube vary from $\sim$2.6 Mtonne
%in \cite{halzenrodrigues2010} (scaling their estimate of
%$\sim$2.4 Mtonne to that of the completed IceCube) to $\sim$3.5 Mtonne
%in \cite{abbasietal2011}.  We use a detection mass of 3.5 Mtonne  
%in our analysis.
Each individual detection component effectively monitors 
an energy-dependent volume of ice surrounding it, with the 
size of the effective volume depending linearly on the 
energy of the interaction products (\citealt{abbasietal2011}).  
Using a cross section-weighted average $\nu_e$ energy of 13 MeV 
for the 15-\Msol case with the LSEOS and the spread of interaction 
product energies, IceCube corresponds to a \abt 900 ktonne 
CCSN neutrino detector.
Although real-time supernova monitoring in IceCube bins data 
in 2-ms time bins, data on the individual photon detections 
are saved in a 90-second window around a putative supernova 
event (\citealt{aartsenetal2013}), allowing for arbitrary binning.
IceCube
currently lacks the ability to measure the energy of the neutrinos it
would detect from a CCSN, so it would be able to measure a light curve, but
not a spectrum for the breakout burst.  For our analysis we assume no
detector threshold for IceCube.  We calculate 
IceCube  will detect 
$\sim$1600 \nue's in a 10-ms period over the
breakout burst at a distance of 10 kpc in the case of no neutrino
oscillations.  For the same situation as just outlined, the number of
detected (original) \nue's in the NH case will be \abt 300 and in the IH
case will be \abt 700.  We also note that because
IceCube does not reproduce neutrinos from CCSN on an event-by-event
basis, it cannot provide any pointing information by itself, as can
Super-K.  However, its large breakout yield may be useful
in a triangulation calculation using multiple detectors.

%IceCube also lacks the ability to discriminate between neutrino
%species.  However, \nue's dominate through the breakout burst peak,
%and in the no-oscillation case the peak in \nue\ luminosity is still
%visible against the \backgrounds.  Because of this, because of the
%potential of using information from other detectors to subtract 
%the \backgrounds\ (especially the \anue\ IBD background), and for
%consistency with the other detectors discussed here, we assume the
%ability to subtract the \backgrounds\ in the no-oscillation case.  As
%with the other detectors, neutrino oscillations complicate things
%and the \backgrounds\ cannot be so easily subtracted in the IH and NH
%cases.
IceCube also lacks the ability to discriminate between neutrino species.  The 
    breakout burst light curve is dominated by \nue's, but the higher-energy 
    \anue\ and \nux's are favored by the energy-dependent effective detection 
    volume and may swamp the \nue\ signal.  Additionally, although the detector 
    background rate is stable, random fluctuations around the average rate (540 Hz 
    per detection component with no dead time and 286 Hz per detection component 
    with a 250-microsecond dead time; \citealt{abbasietal2011}) can also swamp the 
    \nue\ signal.  We relegate a quantitative discussion of the detectability of 
    the \nue\ signal against the backgrounds to Section~\ref{sec:discussion}.


\subsection{Scintillation Detectors}
Scintillation detectors are tanks of hydrocarbon 
scintillators.  They are very similar to water-\cer\
detectors in that 
they employ a proton-rich medium for neutrino detection 
and, as such, are most sensitive  to
\anue's. The final-state electrons and positrons that result from
electron scattering and IBD are detected via their scintillation light
using photomultiplier tubes. 
 Scintillation detectors have a much lower energy detection threshold  
(${\sim}0.2$ MeV, \citealt{lahaetal2014}) than
 water-\cer\ detectors.  In our analysis, 
for the detector efficiency we assume a Heaviside step function 
with step at 0.2 MeV. In addition to detecting
neutrinos through IBD and electron scattering reactions, \nue\
and \anue\ absorption on the carbon nuclei produce detectable
products and the scattering
of all neutrino types on the carbon nuclei can 
in principle  be detected via photon
emission from deexcitation, much as for oxygen in water-\cer\
detectors.  Scintillation detectors can also make a measurement of the
neutrino spectrum by measuring the energies of the final-state
products of the neutrino interactions.

Scintillation detectors have the advantage that electron
scattering 
and IBD events are distinguishable: 99\% of the neutrons formed in IBD
events will quickly (${\sim}0.2$ ms) 
combine with a proton, producing a
2.2-MeV gamma-ray (\citealt{abeetal2008}), which can be detected.  
A coincidence in time and space of an electron/positron signal with a
neutron capture gamma-ray allows the identification of that signal as
a positron.
 In addition, experiments have shown that
scintillation detectors are able to differentiate electrons
and positrons through pulse shape discrimination 
(\citealp{kinoetal2000,francoetal2011}), which
allows for further differentiation between electron scattering
and \anue\ absorption events.  Pulse shape discrimination 
has been demonstrated in active scintillation detectors
(\citealp{abeetal2014,bellinietal2014}) and we anticipate its
continued use in future scintillation detectors.  Because of these
things, in our analysis we assume the ability to tag all IBD's.  For
the no-oscillation case, this corresponds to the \anue\ flux.
Scintillation detectors can detect
\nux\ through $\nu_x + p \rightarrow \nu_x + p$
(\citealp{oberaueretal2005,lahabeacom2014}) and  \nux\ can also, in
principle, be measured via $\nu_x + {^{12}}\textrm{C} \rightarrow  
\nu_x + {^{12}}\textrm{C*}$ (\citealt{ryazhskaya1992}), so we assume in
our analysis that \nux\ can be differentiated from other types.  Thus,
we only care about the \nue\ flux in our analysis for the
no-oscillation case.  Again, complications due to neutrino
oscillations do not permit so straightforward a subtraction of the
\backgrounds\ in the case of oscillations.


Although the exact ratio of carbon to hydrogen varies in 
the scintillators employed in detectors, 
it does not depart too much from a
C$_{n}$H$_{2n}$ stochiometry, which is the chemical form 
assumed in our analysis.

There are currently two scintillation detectors with detection mass
$\sim$1 ktonne: the Kamioka Liquid Scintillator Antineutrino Detector
(KamLAND) in the Kamioka mine in Japan
(\citealt{eguchietal2003}) and the Large Volume Detector (LVD) in 
the Gran Sasso underground laboratory in 
Italy (\citealt{agliettaetal1992}). There are also several smaller detectors.
We calculate 
that a scintillation detector with fiducial mass 1 ktonne will detect 
$\sim$0-1 \nue's in a 10-ms period over the
breakout burst at a distance of 10 kpc in the case of no neutrino
oscillations (even less in the case of neutrino oscillations). 
Because of this small signal,
we do not consider scintillation detectors of this size (and smaller) further.
There is a 14-ktonne
scintillation detector in operation, the NO$\nu$A far
detector in Ash River, Minnesota
(\citealt{pattersonetal2012}), which is located at the surface.  
Because of the
high backgrounds in this detector, we do not consider it.

The Jiangmen Underground Neutrino Observatory (JUNO), currently under
construction\footnote{http://english.ihep.cas.cn/rs/fs/juno0815/PPjuno/201501/
t20150112\_135044.html}, is a 20-ktonne
scintillation detector located in Jiangmen, China (\citealt{li2014}).  We calculate 
that a scintillation detector of JUNO's size will detect 
$\sim$10 \nue's in a 10-ms period over the
breakout burst at a distance of 10 kpc in the case of no neutrino
oscillations.  For the same situation as just outlined, the number of
detected (original) \nue's in the NH case will be \abt 2 and in the IH
case will be \abt 5.  We take the JUNO mass of 20
ktonne as the representative mass for scintillation detectors in our
analysis. 

In summary, \ar40\ detectors have the highest sensitivity to \nue's. 
Scintillation detectors  have the best intrinsic 
particle identification abilities.
Functional and material
considerations make water-\cer\ detectors (including long-string
detectors) less expensive to build with a large detection volume.
\ar40, scintillation, and water-\cer\ detectors are 
all able to measure the energies expected for the
final-state products in a CCSN, while long-string detectors 
are currently unable to measure the energies in that range.  
Table~\ref{tab:numdetections} summarizes, in the case of no neutrino
oscillations, 
our calculations of how many \nue's each of our representative
detectors would be able to detect in a 10-ms period during the
breakout burst at a selection of
distances. Table~\ref{tab:numdetections_NH} shows the same for the NH
case, and  Table~\ref{tab:numdetections_IH} shows the same for the IH case.
Table~\ref{tab:numdetectionsprebreakout}  shows
the same as Table~\ref{tab:numdetections}, in the case
of no neutrino oscillations, for the
pre-shock neutronization peak.  
Table~\ref{tab:numdetectionsprebreakout_NH} shows the same for the NH
case, and  Table~\ref{tab:numdetectionsprebreakout_IH} shows the 
same for the IH case.
The numbers for the pre-shock
neutronization peak are significantly smaller than those of the
breakout burst, because of the lower \nue\ number flux and lower average
\nue\ energy in the pre-shock neutronization peak relative to the
breakout burst peak (Figure~\ref{fig:avgenergy}).  These tables do not
take any \backgrounds\ into account.  In general, the no-oscillation
case causes the largest number of \nue\ detections, while the NH
case causes the smallest number of \nue\ detections, for all
detectors. For the water-\cer, scintillation, and long-string
detectors, this is owing to the smaller electron-scattering cross
section for \nux's as opposed to \nue's (a factor of \abt 6 smaller
for \nuxpart\ and \abt 7 smaller for \nuxanti).
For \ar40\ detectors, this is due to the dominance of the \nue\ CC
absorption channel in the measured signal.
%
\begin{deluxetable}{cccccc}
\tablewidth{0pc}
\tablecolumns{6}
\tablecaption{Approximate Number of $\nu_e$'s Detected in 10-ms
Interval During Breakout in the No-Oscillation Case\label{tab:numdetections}}
\tablehead{
\colhead{Distance} & \colhead{DUNE} & \colhead{Super-K}
& \colhead{Hyper-K} & \colhead{IceCube}
& \colhead{JUNO} \\ \colhead{(kpc)} & \colhead{} & 
\colhead{} & \colhead{} & \colhead{} & \colhead{} }
\startdata
%1 & 12000 & 1000 & 25000 & 160000 & 1100\\
%4 & 740 & 62 & 1600 & 9800 & 68\\
%7 & 240 & 20 & 510 & 3200 & 22\\
%10 & 120 & 10 & 250 & 1600 & 11\\
%20 & 30 & 3 & 62 & 390 & 3
%1 & 12000 & 660 & 16000 & 160000 & 1100\\
%4 & 740 & 41 & 1000 & 10000 & 68\\
%7 & 240 & 13 & 330 & 3300 & 22\\
%10 & 120 & 7 & 160 & 1600 & 11\\
%20 & 30 & 2 & 41 & 400 & 3
1 & 12000 & 660 & 16000 & 44000 & 1100\\
4 & 740 & 41 & 1000 & 2700 & 68\\
7 & 240 & 13 & 330 & 890 & 22\\
10 & 120 & 7 & 160 & 440 & 11\\
20 & 30 & 2 & 41 & 110 & 3
\enddata
\end{deluxetable}

\begin{deluxetable}{cccccc}
\tablewidth{0pc}
\tablecolumns{6}
\tablecaption{Approximate Number of $\nu_e$'s Detected in 10-ms Interval During Breakout in the NH case\label{tab:numdetections_NH}}
\tablehead{
\colhead{Distance} & \colhead{DUNE} & \colhead{Super-K}
& \colhead{Hyper-K} & \colhead{IceCube}
& \colhead{JUNO} \\ \colhead{(kpc)} & \colhead{} & 
\colhead{} & \colhead{} & \colhead{} & \colhead{} }
\startdata
%1 & 230 & 180 & 4500 & 28000 & 220\\
%4 & 14 & 11 & 280 & 1700 & 14\\
%7 & 5 & 4 & 91 & 570 & 5\\
%10 & 2 & 2 & 45 & 280 & 2\\
%20 & 1 & 0 & 11 & 70 & 1
%1 & 230 & 110 & 2800 & 28000 & 220\\
%4 & 14 & 7 & 170 & 1700 & 14\\
%7 & 5 & 2 & 56 & 570 & 5\\
%10 & 2 & 1 & 28 & 280 & 2\\
%20 & 1 & 0 & 7 & 70 & 1
1 & 230 & 110 & 2800 & 5800 & 220\\
4 & 14 & 7 & 170 & 360 & 14\\
7 & 5 & 2 & 56 & 120 & 5\\
10 & 2 & 1 & 28 & 58 & 2\\
20 & 1 & 0 & 7 & 14 & 1
\enddata
\end{deluxetable}


\begin{deluxetable}{cccccc}
\tablewidth{0pc}
\tablecolumns{6}
\tablecaption{Approximate Number of $\nu_e$'s Detected in 10-ms
Interval During Breakout in the IH case\label{tab:numdetections_IH}}
\tablehead{
\colhead{Distance} & \colhead{DUNE} & \colhead{Super-K}
& \colhead{Hyper-K} & \colhead{IceCube}
& \colhead{JUNO} \\ \colhead{(kpc)} & \colhead{} & 
\colhead{} & \colhead{} & \colhead{} & \colhead{} }
\startdata
%1 & 3800 & 440 & 11000 & 69000 & 500\\
%4 & 240 & 27 & 680 & 4300 & 31\\
%7 & 78 & 9 & 220 & 1400 & 10\\
%10 & 38 & 4 & 110 & 690 & 5\\
%20 & 10 & 1 & 27 & 170 & 1
%1 & 3800 & 280 & 6900 & 69000 & 490\\
%4 & 240 & 17 & 430 & 4300 & 31\\
%7 & 78 & 6 & 140 & 1400 & 10\\
%10 & 38 & 3 & 69 & 690 & 5\\
%20 & 10 & 1 & 17 & 170 & 1
1 & 3800 & 280 & 6900 & 18000 & 490\\
4 & 240 & 17 & 430 & 1100 & 31\\
7 & 78 & 6 & 140 & 360 & 10\\
10 & 38 & 3 & 69 & 180 & 5\\
20 & 10 & 1 & 17 & 44 & 1
\enddata
\end{deluxetable}

\begin{deluxetable}{cccccc}
\tablewidth{0pc}
\tablecolumns{6}
\tablecaption{Approximate Number of $\nu_e$'s Detected in 10-ms Interval During Pre-Shock Neutronization Peak in the No-Oscillation Case\label{tab:numdetectionsprebreakout}}
\tablehead{
\colhead{Distance} & \colhead{DUNE} & \colhead{Super-K}
& \colhead{Hyper-K} & \colhead{IceCube}
& \colhead{JUNO} \\ \colhead{(kpc)} & \colhead{} & 
\colhead{} & \colhead{} & \colhead{} & \colhead{} }
\startdata
%1  & 880 & 180 & 4500 & 29000 & 170\\
%4  & 55 & 11 & 280 & 1800 & 11\\
%7  & 18 & 4 & 91 & 590 & 4\\
%10  & 9 & 2 & 45 & 290 & 2\\
%20  & 2 & 0 & 11 & 72 & 0
%1 & 890 & 80 & 2000 & 30000 & 170\\
%4 & 56 & 5 & 120 & 1900 & 11\\
%7 & 18 & 2 & 41 & 600 & 3\\
%10 & 9 & 1 & 20 & 300 & 2\\
%20 & 2 & 0 & 5 & 74 & 0
1 & 890 & 80 & 2000 & 4600 & 170\\
4 & 56 & 5 & 120 & 290 & 11\\
7 & 18 & 2 & 41 & 93 & 3\\
10 & 9 & 1 & 20 & 46 & 2\\
20 & 2 & 0 & 5 & 11 & 0
\enddata
\end{deluxetable}


\begin{deluxetable}{cccccc}
\tablewidth{0pc}
\tablecolumns{6}
\tablecaption{Approximate Number of $\nu_e$'s Detected in 10-ms
Interval During Pre-Breakout Neutronization Peak in the NH
case\label{tab:numdetectionsprebreakout_NH}}
\tablehead{
\colhead{Distance} & \colhead{DUNE} & \colhead{Super-K}
& \colhead{Hyper-K} & \colhead{IceCube}
& \colhead{JUNO} \\ \colhead{(kpc)} & \colhead{} & 
\colhead{} & \colhead{} & \colhead{} & \colhead{} }
\startdata
%1 & 45 & 30 & 750 & 4800 & 29\\
%4 & 3 & 2 & 47 & 300 & 2\\
%7 & 1 & 1 & 15 & 99 & 1\\
%10 & 0 & 0 & 8 & 48 & 0\\
%20 & 0 & 0 & 2 & 12 & 0
%1 & 45 & 12 & 290 & 4800 & 28\\
%4 & 3 & 1 & 18 & 300 & 2\\
%7 & 1 & 0 & 6 & 99 & 1\\
%10 & 0 & 0 & 3 & 48 & 0\\
%20 & 0 & 0 & 1 & 12 & 0
1 & 45 & 12 & 290 & 700 & 28\\
4 & 3 & 1 & 18 & 44 & 2\\
7 & 1 & 0 & 6 & 14 & 1\\
10 & 0 & 0 & 3 & 7 & 0\\
20 & 0 & 0 & 1 & 2 & 0
\enddata
\end{deluxetable}



\begin{deluxetable}{cccccc}
\tablewidth{0pc}
\tablecolumns{6}
\tablecaption{Approximate Number of $\nu_e$'s Detected in 10-ms
Interval During Pre-Breakout Neutronization Peak in the IH
case\label{tab:numdetectionsprebreakout_IH}}
\tablehead{
\colhead{Distance} & \colhead{DUNE} & \colhead{Super-K}
& \colhead{Hyper-K} & \colhead{IceCube}
& \colhead{JUNO} \\ \colhead{(kpc)} & \colhead{} & 
\colhead{} & \colhead{} & \colhead{} & \colhead{} }
\startdata
%1 & 310 & 78 & 1900 & 12000 & 74\\
%4 & 19 & 5 & 120 & 780 & 5\\
%7 & 6 & 2 & 40 & 250 & 2\\
%10 & 3 & 1 & 19 & 120 & 1\\
%20 & 1 & 0 & 5 & 31 & 0
%1 & 310 & 33 & 820 & 12000 & 71\\
%4 & 19 & 2 & 51 & 780 & 4\\
%7 & 6 & 1 & 17 & 250 & 2\\
%10 & 3 & 0 & 8 & 120 & 1\\
%20 & 1 & 0 & 2 & 31 & 0
1 & 310 & 33 & 820 & 1900 & 71\\
4 & 19 & 2 & 51 & 120 & 4\\
7 & 6 & 1 & 17 & 39 & 2\\
10 & 3 & 0 & 8 & 19 & 1\\
20 & 1 & 0 & 2 & 5 & 0
\enddata
\end{deluxetable}


% LocalWords:  lccc 25ex TPCs lahabeacom2014 Antineutrino Jiangmen t20150112
% LocalWords:  cccccc 0pc
