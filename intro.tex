The core-collapse supernova (CCSN) explosion mechanism is a
longstanding unsolved problem in astrophysics. Extant CCSN models
universally indicate that neutrino emission is a key aspect
of a CCSN, with $\sim$99\% of a CCSN's energy carried away by
neutrinos. Such a neutrino signature was confirmed in broad outline 
by the detection of 
neutrinos from SN 1987A (\citealp{bionta1987,hirata1987}).  
Since the events that lead to a CCSN 
are entirely contained 
in the obscured core of the exploding star, the measurement of neutrinos 
(which are able to stream from the core of a star) is vital for testing CCSN theory.  

CCSNe occur as successive stages of nuclear burning build a
degenerate core to the Chandrasekhar limit. At this point, the 
core collapses in on
itself.  During collapse, electron capture causes the core to become neutron-rich.
  The implosion of the core proceeds until the material reaches
densities near nuclear, at which point the implosion
rebounds and produces an outward-propagating shock wave. 
  As the shock propagates, it causes nuclei to dissociate.
Electron capture on the now-free protons produces a large
number of \nue's in the region behind the shock. Initially, the optical
depth seen by these neutrinos prevents their escape from the star, 
but as the shock crosses the \nue-neutrinospheres, the
\nue's produced by the shock (as well as \nue's produced previously 
that have diffused to the neutrinosphere) create a very
luminous (${\sim} 3.5 \times 10^{53}$ erg s$^{-1}$) spike (``breakout
burst'') in the
\nue\ emission, which lasts for ${\sim}10$ milliseconds (ms).  After
this breakout spike,
neutrinos of all types radiate from the proto-neutron
star for ${\sim}10$ or more seconds 
(\citealt{burrowslattimer1986}).  Neutrino oscillations are likely to convert
the \nue's of the breakout burst partially or entirely to other
neutrino flavors (\citealt{mirizzietal2015}).


Prior to the \nue\ breakout burst, 
there is a smaller \nue\ luminosity peak
due to \nue's produced by the neutronization of the core during
 collapse. As the density and temperature of the core increase, the
 opacity increases and eventually these neutrinos are trapped,
 producing a peak and subsequent decrease in luminosity
 (``pre-breakout neutronization peak'').  The peak
 luminosity reached is $\sim$5$\times 10^{52}$ erg s$^{-1}$. 

The breakout burst of a CCSN is a signature
phenomenon that must exist for current CCSN theories to be 
valid. Hence, unambiguous detection of the breakout burst 
in the next Galactic 
CCSN is vital to validating theory and a measurement of 
the properties of the breakout burst
would be important for testing and discriminating between CCSN models. Since 
Galactic CCSNe occur at a rate of $\sim$3 per century 
(\citealt{adamsetal2013}),  to take advantage of the next
Galactic CCSN we must constantly be ready to
take data.  The Supernova Early Warning System, SNEWS, provides such
constant vigilance
 (\citealp{antoniolietal2004,scholberg2008}).

The very property of neutrinos that allows them to stream through the
stellar mantle also makes them very difficult to detect. Only 19
neutrinos were detected in the Kamiokande II and IMB detectors from SN
1987A (\citealp{bionta1987,hirata1987}). This was sufficient to confirm
general details of CCSN theory, but lacked sufficient discriminating
power to 
truly differentiate between models, as well as lacked detail to see
the \nue\ breakout burst.  The current generation of neutrino
detectors promises much larger integral signals for a Galactic CCSN,
but will they be adequately sensitive to detect and characterize the
inaugural breakout burst?
This work seeks to answer this question in the context of
current and likely future neutrino detectors.

The CCSN
models used for our analysis are introduced in
Section~\ref{sec:model}, and the expected evolution of
the detected breakout burst signal is discussed in
Section~\ref{sec:signalevolution}.
The various classes of 
neutrino detectors are
discussed in Section~\ref{sec:detection}. The method of our analysis is
outlined in Section~\ref{sec:method} and the results
of our analysis are explored in Section~\ref{sec:discussion} (which
includes results from the no-oscillation case, as well as
those due to the normal and inverse neutrino mass hierarchies).
We then conclude in 
Section~\ref{sec:conclusion}.
