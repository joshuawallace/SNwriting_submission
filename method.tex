Our goal in this paper is to determine, for highlighted detectors, 
for each SN
model, and over a range of distances, how well one can expect to measure 
the shape and features of the \nue\ breakout light curve, as well as
examine how well different CCSN models can be discriminated using this
light curve. To do this, we use
Equation~(\ref{eq:Nint}) to calculate the total expected number of
neutrino interactions in a given detector for a given distance over a
time range that includes 
the breakout burst peak.  A cumulative distribution function of
the arrival times of the detected neutrinos is also calculated.  We
then perform a Monte Carlo sampling of observations in the given
detector, using the cumulative distribution function of detection
times and the calculated total expected number of neutrino
interactions to create simulated realizations of individual
observations.  
The cross sections used in our analysis are detailed in
the Appendix.
The time range we use in the Monte Carlo sampling is
larger than that used for our subsequent analysis, so that the total
number of neutrinos actually used in the analysis is allowed to
fluctuate randomly, instead of being set by our calculated total expected
number of neutrino interactions.   A collection of $5\times10^4$ 
sample observations are thus assembled for each detector and supernova
distance.  The simulated data
are binned in time (in time bins of 1-ms width) 
and standard deviations for each time bin are
calculated using the values across the $5\times10^4$ different
simulated observations.
These standard deviations are then applied to each of the individual simulated
observations for the purposes of fitting our analytic equations to the
simulated data.  For simplicity, we assume symmetric errors in each
time bin of the light curve.  Since the distribution of the number 
of detections in
each time bin is expected to be Poissonian (an expectation that we
verified in our simulated data), the distribution is asymmetric.  
However, in the larger detectors
and at smaller distances a sufficient number of neutrinos will be
detected in each time bin for the Poissonian distribution to approach a
symmetric Gaussian, so our assumption holds in these cases.  For
smaller detectors and larger distances, with fewer detections expected
in each time bin, our assumption is not valid, but we hold to it both
for simplicity and also because the sparse data expected with smaller
detectors and at larger distances will themselves create large errors in the
distribution of fits to our simulated observations, larger than those
which could be corrected by providing asymmetric errors in each time bin.

As a specific example of a calculation, 
Figure~\ref{fig:hyperkhistogram} shows the results of one of the $5\times10^4$
realizations of a CCSN detection in Hyper-K for each of the distances 4, 7, and
10 kpc for the 15-\Msol\ \ls\ progenitor model. 


After the simulated observations are assembled, 
Equation~(\ref{eq:analytic}) is
then fit to the resultant number luminosity histograms (with error bars)
using the \texttt{curve\_fit()} function
of SciPy.  
\texttt{curve\_fit()} implements
the Levenberg-Marquardt algorithm to fit data to a function with arbitrary
parameters.  The function we fit is derived from 
Equation~(\ref{eq:Nint}), as follows.  First, the energy luminosity spectrum is
converted to number luminosity spectrum, $dL^{n}_{\nu}/dE_\nu$,
\beq
\frac{dL^{n}_{\nu}(E_\nu,t)}{dE_\nu} = \frac{dL_\nu(E_\nu,t)}{dE_\nu}\frac{1}{E_\nu}.
\eeq
The number luminosity spectrum is then normalized,
\beq
\label{eq:numlumnorm}
\frac{dL^{n \prime}_{\nu}}{dE_{\nu}} = \frac{dL^{n}_{\nu}/dE_{\nu}}{\int dL^{n}_{\nu}/dE_{\nu}~~dE_{\nu}}.
\eeq
Since
\beq
L^n_\nu(t) = \int \frac{dL^n_\nu(E_\nu,t)}{dE_\nu} dE_\nu,
\eeq
we can rearrange Equation~(\ref{eq:numlumnorm}) to get
\beq
\label{eq:substitution}
\frac{dL^{n}_{\nu}(E_\nu,t)}{dE_\nu} = L^n_\nu(E_\nu,t)\frac{dL^{n \prime}_{\nu}}{dE_{\nu}}.
\eeq
 Equation~(\ref{eq:substitution}) can be substituted into 
Equation~(\ref{eq:Nint}) to obtain
\begin{equation}
\label{eq:blackbox}
N_{det}(t) =  \frac{N_{t}}{4\pi D^2}
L^n_\nu(t,\boldsymbol{p})
\int 
\frac{dL^{n \prime}_{\nu}}{dE_{\nu}}\sigma(E_\nu)\epsilon(E_e)dE_\nu,
\end{equation}
where $L^n_\nu$ is given by
Equation~(\ref{eq:analytic}), the superscript $n$
specifies the use of number luminosity instead of energy luminosity, 
$\boldsymbol{p}$ represents the parameter values being used in 
Equation~(\ref{eq:analytic}) or~(\ref{eq:smallpeak}), $dN'/dE_\nu$ is the
normalized number spectrum, and the other symbols have the same
meanings as in Equation~(\ref{eq:Nint}). To use
Equation~\ref{eq:blackbox}, the distance to the supernova must be
known.  If the supernova is visible, an independent measurement of $D$
can be made.  If the supernova is obscured, the distance will likely
have to be estimated from the neutrino signal.  In our analysis, we
assume knowledge of the distance. In this analysis, we have the
advantage of knowing the energy spectrum from our models, and that is
the spectrum which is used in Equation~\ref{eq:blackbox} for our
analysis.  An actual detection might entail the measurement of the
energy spectrum of the neutrinos, and will likely have an additional
function and parameterization that will be used to fit the spectrum.
We do not perform such a full analysis in this work, but in principle
it would be straightforward. Figures~\ref{fig:energyspectrum}
and \ref{fig:avgenergy} show that the energy distribution and average
energy do not vary appreciably over the duration of the breakout
burst, so even something as simple as assuming a constant spectrum
through the breakout burst would be reasonable.  
Therefore, measurements of the spectrum could be integrated over
the time of the breakout burst to provide higher statistics in
measuring the energy spectrum than measuring a time-dependent
energy spectrum.  
For a given detector (which has
multiple detection channels), Equation~(\ref{eq:blackbox}) can be applied
to all the interaction channels and the results summed together.
The equation we give to
the \texttt{curve\_fit()} function to fit the simulated data is
Equation~(\ref{eq:blackbox}), while the parameters that are being used
in the fitting algorithm are those of the intrinsic number
luminosity. 
The Levenberg-Marquardt algorithm requires an initial guess for the
parameters, for which we provide the values from
Table~\ref{tab:numpeakfit}.

After the best fit fitting parameters are calculated, the physical
parameters are derived from the fit.  
We emphasize again that it is not the fitting parameters, but rather the
physical parameters that are important to
our analysis.  For the main breakout burst peak, the physical parameters
calculated are:
 maximum number luminosity of breakout burst
($L^n_{\nu_e,\mathrm{max}}$),
the time of maximum luminosity ($t_{\mathrm{max}}$), the width of the
peak ($w$),
the rise time ($t_{\mathrm{rise},1/2}$), and the fall time
($t_{\mathrm{fall},1/2}$).  
%For the pre-shock
%neutronization peak, the physical parameters calculated are the
%maximum number luminosity of the pre-shock peak
%($L^n_{\nu_e,\mathrm{max,pre}}$) and the time of this maximum luminosity 
%($t_{\mathrm{max,pre}}$).

