
The models used in this study were produced using F{\sc{ornax}} (J. Dolence 
\& A. Burrows 2015, in preparation).
F{\sc{ornax}} is a code written in generally covariant form for 
multidimensional,
self-gravitating, radiation hydrodynamics that is second-order accurate in
space and time and was designed from scratch with the CCSN 
problem in mind. The code solves the equations of compressible
hydrodynamics with an arbitrary equation of state (EOS), coupled to the
multigroup two-moment equations for neutrino transport. The hydrodynamics
in F{\sc{ornax}} is based on a directionally unsplit Godunov-type finite-volume
method. Fluxes at cell faces are computed with the fast and accurate HLLC
approximate Riemann solver, with left and right states given by limited
parabolic reconstructions of the underlying volume-averaged states. The
multigroup two-moment equations for neutrino transport are formulated in
the comoving frame and include all terms to $\mathcal{O}$($v/c$). 
The moment hierarchy
is closed with the ``M1" model. F{\sc{ornax}} adopts a Godunov-type approach for
treating the transport-related divergence terms that requires that we
solve a generalized Riemann problem at each face. All of the these
transport-related terms are treated explicitly in time. After core bounce,
the fastest hydrodynamic signal speeds in the CCSN 
problem are within a factor of a few of the speed of light, so explicit
time integration is not only simpler and generally more accurate but it is
also faster than globally coupled time implicit transport solves that are
typically employed in radiation hydrodynamics. The source and sink terms
that transfer energy between the radiation and the gas are operator-split
and treated implicitly. These terms are purely local to each cell and do
not introduce any global coupling.

 Progenitors with mass of 12, 15, 20, and
25 \Msol\ from \cite{woosleyheger2007} 
were simulated in 1D with the Lattimer--Swesty high-density
nuclear EOS with $K=220$ MeV (\ls;
\citealt{lattimerswesty1991}), while a progenitor with mass 15
\Msol\ was simulated using the Shen nuclear EOS (\shen;
\citealp{shenetaljul1998,shenetalnov1998}).  This set of models is
chosen to give a good span of potential progenitors for an actual
CCSN. 
 Since in this study we focus on the breakout burst, the full ${\sim}10$ s of
core neutrino cooling is not important for our study and is, thus, not
discussed further.

Figure~\ref{fig:lumallt} shows the early (unoscillated) neutrino light curve for the
neutrino species in the models: \nue, \anue, and \nux\ (which represents
$\nu_\mu$, $\nu_\tau$, and their antiparticles collectively), with
$L_{\nu_i}$ representing the $\nu_i$ energy luminosity. In this work,
we use \nuxpart\ to refer to $\nu_\mu$ and $\nu_\tau$ and use \nuxanti\
to refer to $\bar\nu_\mu$ and $\bar\nu_\tau$.
Additionally, $L_{\nu_x}$ refers to the luminosity due to all of
the \nux's together, while  $L_{\nu_{\mu,\nobreak \tau}}$ and
$L_{\bar\nu_{\mu,\nobreak \tau}}$ refer to the luminosity due to just
one of their constituent species.
The initial peak in \nue\ luminosity
from the preshock neutronization of the core is evident, followed
by the much steeper rise and
larger peak in \nue\ luminosity due to the breakout burst. After the 
breakout burst, the \anue\ and
\nux\ luminosities rise to nearly constant values.  The \nue\ luminosity
then falls to a comparably constant value.  In studying the
breakout burst, \anue\ and \nux\ serve as a background 
to the \nue\ signal we want to detect.

Figure~\ref{fig:nuelumt} shows the unoscillated-\nue\ number luminosity as a function of
time during breakout for the various models. The light curves are
centered around their maximum values. The \shen\ model has the largest
peak luminosity,
while the \ls\ models are grouped together at a slightly lower peak
luminosity. 
Results from \cite{sullivanetal2015} suggest that the higher peak 
\nue\ luminosity and smaller light curve width associated with the \shen\ 
are due to its smaller electron-capture rate (relative to that 
using the \ls), particularly on infall.  The different 
electron-capture rates are due to different free proton 
abundances and nucleon chemical potentials, the latter of which 
also affect the stimulated absorption correction of electron capture.
This tight grouping of peak luminosities for all progenitors, 
if in fact real,
can be used as a standard candle to determine the distance to a
supernova (SN).  \cite{kachelriessetal2005} suggest that an SN at 10 kpc
could have its distance determined to a precision of \abt 5\%, if these
theoretical predictions play out.

The goal of our analysis here is to examine, for a given
detector, for a given SN distance, how well the various properties of
this breakout burst can be constrained based on expected neutrino
signals, as well as determine whether our different CCSN models can
be differentiated based on a measurement of the breakout burst.  To
quantify various properties of the breakout burst, we define the
following physical parameters: 
the maximum number 
luminosity of breakout burst
($L^n_{\nu_e,\mathrm{max}}$),
the time of maximum luminosity ($t_{\mathrm{max}}$), the width of the
breakout burst peak ($w$, calculated using the FWHM), 
the rise time ($t_{\mathrm{rise},1/2}$, calculated
using the width left of peak at half-maximum), and the fall time
($t_{\mathrm{fall},1/2}$, calculated using
the width right of peak at half-maximum).  For the preshock
neutronization peak we define the following physical parameters: the
maximum number luminosity of the preshock peak
($L^n_{\nu_e,\mathrm{max,pre}}$) and the time of this maximum luminosity 
($t_{\mathrm{max,pre}}$).
Table~\ref{tab:numpeakfit_realparm} shows the values the physical
parameters of the breakout burst take in our  number luminosity light
curves, and Table~\ref{tab:smallbumpphysicalparms} shows the values
the physical parameters of the preshock neutronization peak take in
the same curves.
The fact that  $t_{\mathrm{max}}=0$ for all models is to be
expected, since the zero point in time for all the models is defined
to be the time of maximum luminosity.
\begin{deluxetable}{lccccc}
%\begin{deluxetable*}{lccccc}
\tablewidth{\linewidth}
\tablecolumns{7}
\tablecaption{Physical Parameters of the Breakout Peak Fit to the Number Luminosity, L$^n_{\nu_e}$\label{tab:numpeakfit_realparm}}
\tablehead{
\colhead{Model} & \multicolumn{1}{c}{$L^n_{\nu_e,\mathrm{max}}$} & \colhead{$t_{\mathrm{max}}$}
& \colhead{$w$} & \colhead{$t_{\mathrm{rise},1/2}$}
& \colhead{$t_{\mathrm{fall},1/2}$}
%& \colhead{$L^n_{\nu_e,\mathrm{max,pre}}$}
\\ [0.35ex] %& \colhead{$t_{\mathrm{max,pre}}$}\\ [0.35ex]
\colhead{(\Msol)} &  \colhead{$(10^{58} $ s$^{-1})$} &  \colhead{(ms)} &  \colhead{(ms)}
&  \colhead{(ms)}
&  \colhead{(ms)} }%& \colhead{$(10^{58} $ s$^{-1})$} & \colhead{(ms)}}
\startdata
12 & 1.97 & 0.00 & 8.85 & 2.23 & 6.62 \\%& 0.552 & -6.51\\
15 & 2.01 & 0.00 & 9.29 & 2.24 & 7.06 \\%& 0.577 & -6.48\\
15 S & 2.23 & 0.00 & 8.71 & 1.70 & 7.01 \\%& 0.629 & -5.73\\
20 & 2.03 & 0.00 & 10.2 & 2.22 & 7.96 \\%& 0.597 & -6.43\\
25 & 2.02 & 0.00 & 10.4 & 2.28 & 8.10 %& 0.604 & -6.48
\enddata
%\end{deluxetable*}
\end{deluxetable}
%
\begin{deluxetable}{lcc}
\tablewidth{0pc}
\tablecolumns{5}
\tablecaption{Physical Parameters of the Preshock Neutronization Peak Fit to the Number Luminosity, L$^n_{\nu_e}$\label{tab:smallbumpphysicalparms}}
\tablehead{
\colhead{Model} & \multicolumn{1}{c}{$L^n_{\nu_e,\mathrm{max,pre}}$} & \colhead{$t_{\mathrm{max,pre}}$} \\
  \colhead{(\Msol)} & \colhead{(10$^{58}$ s$^{-1}$)} &\colhead{(ms)} }
\startdata
12 & 0.552 & -6.51\\
15 &  0.577 & -6.48\\
15 S & 0.629 & -5.73\\
20 & 0.597 & -6.43\\
25 & 0.604 & -6.48
\enddata
\end{deluxetable}

We construct an analytic
model for the main breakout peak of our numerical light curves similar to 
equation (10) of \cite{burrowsmazurek1983}. This analytic model will
later be used to fit simulated observations of the \nue\ breakout
burst light curve to measure various physical parameters of the
breakout burst.  The function we use to 
fit the main peak is
\begin{equation}
\label{eq:analytic}
L(t) = \frac{A}{((t-t_{c})/[\textrm{ms}])^\alpha} \exp\left[ -\left(\frac{b}{t-t_{c}}\right)^\beta\right] + L_{\textrm{base}},
\end{equation}
where $A$ is a scaling parameter with units of (for energy luminosity) erg
s$^{-1}$ or (for number luminosity) s$^{-1}$, $b$ is a
width parameter, $t$ is the time since the maximum 
\nue\ number luminosity, $t_{c}$
is a number used to center the fit appropriately,
$\alpha$ and
$\beta$ are exponents similar to those used in
\cite{burrowsmazurek1983}, 
and $L_{\textrm{base}}$ (in units of erg s$^{-1}$ for energy luminosity or s$^{-1}$ for
number luminosity) is used to set the floor of the fit.  Since the
intent of Equation~(\ref{eq:analytic}) is to provide a fit to our models
for subsequent analysis, we do not spend much time in this work 
examining its physical significance. We refer readers to
\cite{burrowsmazurek1983} for a motivation of its functional form.  
We do note that the floor set by $L_{\textrm{base}}$ can be thought of as 
the level 
to which the \nue\ luminosity decays as the luminosity source
transitions from the electron capture that dominates the breakout
burst to the accretion, deleptonization, and cooling phases of the proto-neutron star. 


\begin{deluxetable*}{lcccccc}
%\begin{deluxetable}{lcccccc}
\tablewidth{\linewidth}
\tablecolumns{7}
\tablecaption{Breakout Peak Fit to the Energy Luminosity, L$_{\nu_e}$\label{tab:peakfit}}
\tablehead{
\colhead{Model} & \multicolumn{1}{c}{$A$} & \colhead{$b$}
& \colhead{$t_c$} & \colhead{$\alpha$}
& \colhead{$\beta$} & \colhead{$L_{\textrm{base}}$} \\ 
\colhead{(\Msol)} &  \colhead{$(10^{57} $ erg s$^{-1})$} &
\colhead{(ms)} &  \colhead{(ms)}
&  \colhead{} &  \colhead{}
&  \colhead{$(10^{53} $ erg s$^{-1})$}  }
\startdata
12 & 0.620 & 12.2 & -4.69 & 3.06 & 1.09 & 0.360\\
15 & 1.52 & 20.6 & -4.33 & 3.20 & 0.849 & 0.418\\
15 S & 1.00e14 & 6.65e8 & -2.49 & 6.26 & 0.182 & 0.345\\
20 & 1.00e14 & 2.33e7 & -3.47 & 6.96 & 0.220 & 0.501\\
25 & 2.33e5 & 7090 & -3.69 & 4.91 & 0.349 & 0.511
\enddata
\end{deluxetable*}
%\end{deluxetable}
%
Figure~\ref{fig:fit}
displays the fits of
Equation~(\ref{eq:analytic}) to our numerical models.  The fits 
match the numerical models quite well.    The
analytic fitting parameters to the energy luminosity of  all the
numerical models are given in 
Table~\ref{tab:peakfit}, and the  fitting parameters to the number 
luminosity of  all the
numerical models are given in 
Table~\ref{tab:numpeakfit}.  The \shen\ model is represented by ``S'' in the
``Model'' column; those models not marked with an ``S'' use
the \ls. This is a convention we take throughout this work. 
Two of the parameters ($A$ and $b$) show
variation over many orders of magnitude across the fits of the
different models.  Because of this, we set a maximum value for $A$,
which equates to (for energy luminosity) $10^{71}$ erg s$^{-1}$ and
(for number luminosity) $10^{71}$ s$^{-1}$.  The choice of this
particular maximum value is arbitrary.  It is based on our experience
with the behavior of the fits with unbounded parameters.  Since the fits are so
good, and since we care about the physical parameters that 
are derived from the fits rather than the fit parameters themselves, 
we see no harm in doing this.  Additionally, we 
force $\alpha$, $\beta$,
$L_{\textrm{base}}$, and $A$ to be positive.  We wish to emphasize that
although the parameters of Table~\ref{tab:peakfit} vary to a 
large degree between models, the important consideration in our analysis is
not the parameters themselves, but the characteristics of the fit they
provide (i.e. the physical parameters introduced previously and
shown in Table~\ref{tab:numpeakfit_realparm}), so the
fidelity of the fits in representing the numerical models is far more
important than the values the fit parameters
take. A reason for the large variation in the 
best-fit values of
$A$ and $b$ is the very large, positive covariance between these two
parameters.  Since the specific values of these parameters are not
important, we do not focus on the covariance
data in this work. 

\begin{deluxetable}{lcccccc}
\tablewidth{\linewidth}
\tablecolumns{7}
\tablecaption{Breakout Peak Fit to the Number Luminosity, L$^n_{\nu_e}$\label{tab:numpeakfit}}
\tablehead{
\colhead{Model} & \multicolumn{1}{c}{$A$} & \colhead{$b$}
& \colhead{$t_c$} & \colhead{$\alpha$}
& \colhead{$\beta$} & \colhead{$L^n_{base}$} \\ [0.35ex]
%& \colhead{$\beta$} & \colhead{$L^n_{base}$} \\ [0.1ex]
\colhead{(\Msol)} &  \colhead{$(10^{61} $ s$^{-1})$} &  \colhead{(ms)} &  \colhead{(ms)}
&  \colhead{} &  \colhead{}
&  \colhead{$(10^{57} $ s$^{-1})$}  }
\startdata
12 & 3.29 & 16.3 & -4.30 & 2.94 & 0.892 & 2.88\\
15 & 39.7 & 60.1 & -3.90 & 3.39 & 0.621 & 3.35\\
15 S & 1.00e10 & 2.71e7 & -2.38 & 4.95 & 0.198 & 2.76\\
20 & 1.00e10 & 8.44e5 & -3.31 & 5.68 & 0.251 & 3.92\\
25 & 5.18e9 & 4.89e5 & -3.41 & 5.67 & 0.260 & 4.06
\enddata
\end{deluxetable}



Additionally, we fit the preshock neutronization peak with a
modified lognormal curve,
\begin{equation}
\label{eq:smallpeak}
L(t) =
C \exp\left(\frac{-(\ln(t' -\theta)-\mu)^2}
{2\sigma^2}\right)(t'-\theta)^{-1},
\end{equation}
where $t'=-t/$[ms], $C$ is a scaling factor, and all the other
parameters have their
usual meaning in relation to the lognormal distribution: $\sigma$ is
the standard deviation, $\theta$ is the location parameter, and $\mu$
is the median of the distribution.  The best-fit values to the
preshock  energy
luminosity for the various models are shown in
Table~\ref{tab:smallbumpfit}, and  Table~\ref{tab:numsmallbumpfit} shows the
best-fit values for the preshock 
number luminosity. Figure~\ref{fig:fit}
displays the fits of
Equation~(\ref{eq:smallpeak}) to our numerical models. 
The physical parameters we define for the 
preshock neutronization peak are (as introduced previously) the
maximum number luminosity of the preshock peak
($L^n_{\nu_e,\mathrm{max,pre}}$) and the time of this maximum luminosity 
($t_{\mathrm{max,pre}}$).  The model values of these parameters are
shown in Table~\ref{tab:smallbumpphysicalparms}. It is the fits
themselves (Equations~\ref{eq:analytic} and~\ref{eq:smallpeak}), with
the appropriate fitting parameters, that we use as our baseline models for our
analysis over the time ranges fitted by them, not the numerical data 
from F{\sc{ornax}}.  The numerical data are used in the time ranges
not fitted by Equations~\ref{eq:analytic} and~\ref{eq:smallpeak}
(i.e., the time ranges before to the pre-breakout neutronization peak
and after the breakout burst).


\begin{deluxetable}{lcccc}
\tablewidth{0pc}
\tablecolumns{5}
\tablecaption{Preshock Neutronization Peak Fit to the Energy Luminosity, L$_{\nu_e}$\label{tab:smallbumpfit}}
\tablehead{
\colhead{Model} & \multicolumn{1}{c}{$C$} & \colhead{$\sigma$}
& \colhead{$\theta$} & \colhead{$\mu$} \\
  \colhead{(\Msol)} & \colhead{(10$^{53}$ erg s$^{-1}$)} &\colhead{} &\colhead{} &\colhead{}  }
\startdata
12 & 3.93 & 0.822 & 1.96 & 2.08\\
15 & 4.28 & 0.809 & 1.79 & 2.12\\
15 S & 4.78 & 0.790 & 0.925 & 2.08\\
20 & 5.17 & 0.708 & 0.653 & 2.20\\
25 & 5.41 & 0.682 & 0.404 & 2.22
\enddata
\end{deluxetable}


\begin{deluxetable}{lcccc}
\tablewidth{0pc}
\tablecolumns{5}
\tablecaption{Preshock Neutronization Peak Fit to the Number Luminosity, L$^n_{\nu_e}$\label{tab:numsmallbumpfit}}
\tablehead{
\colhead{Model} & \multicolumn{1}{c}{$C$} & \colhead{$\sigma$}
& \colhead{$\theta$} & \colhead{$\mu$} \\
  \colhead{(\Msol)} & \colhead{(10$^{58}$ s$^{-1}$)} &\colhead{} &\colhead{} &\colhead{}  }
\startdata
12 & 3.74 & 0.834 & 1.72 & 2.26\\
15 & 3.91 & 0.857 & 1.79 & 2.28\\
15 S & 4.32 & 0.817 & 0.809 & 2.26\\
20 & 4.63 & 0.754 & 0.596 & 2.33\\
25 & 4.40 & 0.807 & 1.22 & 2.31
\enddata
\end{deluxetable}

The energy spectrum of all neutrino 
types varies during the shock breakout.  Figure~\ref{fig:energyspectrum} 
shows the \nue\ energy spectrum as a
function of time through the breakout burst for a specific model, 
the 15-\Msol\ \ls\ model, derived using F{\sc{ornax}}.
We define the average neutrino energy $\overline{E}_{\nu_i}$ as
\beq
\label{eq:averageenergy}
\overline{E}_{\nu_i}(t) \equiv \frac{\int L_{\nu_i}(E_{\nu_i},t)~dE_\nu}{\int L_{\nu_i}(E_{\nu_i},t)/E_{\nu_i}~dE_{\nu_i}}.
\eeq
  Figure~\ref{fig:avgenergy} shows the average neutrino energy as
a function of time for all the models (with the zero point in time set
to be the time of maximum number luminosity). 
All the models show the same
general behavior. 
$\overline{E}_{\nu_e}$ increases from the onset
of the breakout burst, peaks near \tmax, and then
decays slightly.  However, during the breakout, $\overline{E}_{\nu_e}$ 
does not change radically and is similar from model to
model.
 There is a slight trend for the average \nue\ energy during
breakout to be slightly higher for the lower-mass progenitors.  In
addition, the \shen\ results in a slightly higher mean \nue\ energy than
the \ls.  In all cases, the average energy peaks $\lesssim$1 ms later than
the maximum number luminosity.


% LocalWords:  ornax lccccccc 35ex lcccccc 00e14 65e8 33e7 33e5 00e10 71e7 44e5
% LocalWords:  18e9 89e5 lcccc 0pc
