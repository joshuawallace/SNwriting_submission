
We first discuss the results obtained without neutrino oscillations
taken into account, followed by the results expected based on the
neutrino oscillation scenarios due to the NH and IH. For
the purpose of this analysis, we take one model (15 \Msol, \ls) as an
example. 
Throughout this Section (and this work), we use the 95\% uncertainties
as the basis for our discussion.

\subsection{Results Without Neutrino Oscillations}

We first consider the case of no neutrino oscillations.  While this is
not likely to be the case, it provides a good baseline for
quantifying the capabilities of neutrino detectors in measuring the
properties of the \nue\ breakout burst.  This
is for two reasons. The first is that the no-oscillation case 
represents the case with
the largest detectable \nue\ flux. Since \nuxpart's to which the \nue's
oscillate, either partially (in the IH) or entirely (in the NH), 
have systematically smaller interaction cross sections
than do \nue's in the detectors of our analysis. The 
second reason is that 
any oscillations of \nuxpart's to \nue's or \nuxanti's to \anue's opens 
interaction cross sections to these species that are larger than those 
they otherwise could access, and so the \background\ levels increase.
 Thus, the no-oscillation case represents the
maximum performance level of the detectors in our analysis in terms of
maximizing the \nue\ signal and minimizing \backgrounds.

IceCube's 540 Hz average background rate per detection component, when 
    multiplied by the total number of detection components (5160), gives a total 
    background rate of 2786 ms$^{-1}$.  Assuming Poissonian noise, the fluctuations
    on this rate are $\sqrt{2786} = 53$ ms$^{-1}$.  As can be seen from Table~\ref{tab:numdetections}, the 
    expected $\nu_e$ detection rate through the breakout burst peak for a CCSN at 
    10 kpc in the case of no oscillations is 44 ms$^{-1}$, smaller than the expected fluctuations in the detector 
    background rate.  The \nue\ count rate is only lower for both
    of the two oscillation scenarios.  Smaller CCSNe distances will provide a higher count rate,
    but even a CCSN at 7 kpc will have a count rate of only 89 ms$^{-1}$, somewhat 
    larger but still comparable to the detector background fluctuations.  Even with 
    introducing a 250-microsecond dead time to lower the background rate to 
    286 Hz (which leads to a \abt 13\% dead time total; \citealt{abbasietal2011}), the 
    Poissonian fluctuations on the detector background rate are 37 ms$^{-1}$, as 
    compared with the reduced 38 ms$^{-1}$ \nue\ signal rate for CCSNe at 10 
    kpc and 77 ms$^{-1}$ for CCSNe at 7 kpc.  Even if a CCSN was sufficiently 
    close to distinguish a signal against the detector background fluctuations, there 
    is still the issue of extracting the \nue\ signal from the
    \backgrounds, 
which our calculations show begin dominating in the first few ms 
    after the peak \nue\ luminosity.  In light of all this, it is doubtful that IceCube
    will be able to extract a meaningful signal of the breakout burst in a Galactic
    CCSN and we do not consider IceCube further in our quantitative analysis of the
    performance of our highlighted neutrino detectors in measuring the properties of the
    breakout burst.


For the figures and tables in this section, 
we take one model as an example model (15 \Msol, \ls).


\subsubsection{Physical Parameter Probability Distribution Functions}

We show here the PDFs of the
physical parameters derived from our simulated observations for each
detector we consider in our analysis without neutrino oscillations
taken into account.  Each detector is representative
of a given detector type.  In order of presentation in this section,
they are: Super-K and
Hyper-K, representing water-\cer\ detectors; the 40-ktonne DUNE far
detector (hereafter referred to simply as ``DUNE''), 
for \ar40\ detectors; and JUNO, for scintillation detectors.

%Figure~\ref{fig:icecubephysicalparms_hist} shows the PDFs of the
%physical parameters derived from the fits to the simulated
%observations for IceCube in the no-oscillation case. 
%For comparison purposes, the horizontal
%scales of Figure~\ref{fig:icecubephysicalparms_hist} 
%are the same as for the PDF's of the
%corresponding quantities in the next several figures.  In the
%no-oscillation case, IceCube would be able
%to constrain these parameters quite well at both 4 and 10 kpc relative
%to the other detectors examined in our study, as will be shown. 

Figure~\ref{fig:superkphysicalparms_hist} shows the PDFs of the
physical parameters derived from the fits to the simulated
observations for Super-K in the no-oscillation case.  
It is important to
note is that, in the no-oscillation case, 
the distributions of \lmax\ and \tmax\ are relatively
symmetric about the model value (vertical green line), 
while the distributions of $w$, \trise, and \tfall\ are 
asymmetric, being skewed to higher values.  This causes the mode of the
distributions for these parameters to occur at a value
smaller than the model value.

Figure~\ref{fig:hyperkphysicalparms_hist} shows the PDFs of the
physical parameters derived from the fits to the simulated
observations for Hyper-K in the no-oscillation case.  
The widths of the distributions are less than 
those of Super-K, consistent with Hyper-K's mass being
greater than Super-K's.  The asymmetries in the
distributions of 
$w$, \trise, and \tfall\ that are seen in
Super-K are also exhibited in Hyper-K, though to a lesser degree.

Figure~\ref{fig:dunephysicalparms_hist} shows the PDF of the
physical parameters derived from the fits to the simulated
observations for DUNE in the no-oscillation case.  The widths of the
distributions are less than 
those of Super-K, but comparable to Hyper-K, again consistent with the number of
detected neutrinos expected in DUNE relative to both of these
detectors.  The asymmetries of $w$, \trise, and \tfall\ seen in
Super-K are present in DUNE.  

Figure~\ref{fig:junophysicalparms_hist} shows the PDF of the
physical parameters derived from the fits to the simulated
observations for JUNO in the no-oscillation case.  


For each parameter, a PDF is calculated and the 95\%
confidence values are calculated.  By repeating this process over a
set of distances for the detectors highlighted in this study, we can
determine, for a given detector and SN model, how well the various
features of the breakout burst light curve can be determined as a
function of distance.


\subsubsection{Detector Performance for Measuring \lmax}
Figure~\ref{fig:15maxvalfuncD} shows the 95\% uncertainty in
measuring \lmax\ (the maximum value of the breakout burst luminosity)
in the no-oscillation case as a function of distance for the detectors in our
analysis for the 15-\Msol\ model employing the \ls. 
Table~\ref{tab:maxvalpercenterrortable} shows, for each representative
detector and as a function of distance, in the no-oscillation case, 
the mode of the PDF obtained
in each case for \lmax\ as well as the percent
errors associated with 95\% uncertainties.  For both
Figure~\ref{fig:15maxvalfuncD} and
Table~\ref{tab:maxvalpercenterrortable},  when the uncertainty values for a
  specific detector get either
  too large or too small relative to the model value, we no longer
  represent 
  the uncertainty at that distance and greater distances.  For
  Figure~\ref{fig:15maxvalfuncD} (and subsequent figures in this
  Section), 
this simply means the data are no
  longer plotted for these distances.  For
  Table~\ref{tab:maxvalpercenterrortable} (and subsequent tables in
  this Section), the missing data are
  represented with ellipses.
The uncertainty values for 
Super-K and JUNO were cut off
  at 7 kpc if the previous criteria were not met at 7 kpc because of
  the small number of events for a supernova beyond that
  distance.
The mode is obtained
by fitting a Gaussian curve to the peak of the PDF.
 For this and all the other parameters, there is a clear
hierarchy in the detectors' abilities to precisely measure the
parameters.  Detectors expected to detect a larger number of \nue's in
a breakout burst in the no-oscillation case 
(due to their larger mass and/or use of a more
\nue-sensitive detection medium) can more accurately measure
the physical parameters for a given supernova distance.  
Specifically, Super-K
and JUNO's smaller detection volumes and cross sections make them least
likely to make an accurate measurement in the no-oscillation, with the accuracy of  
Hyper-K and DUNE being greater, owing to their larger detection
volumes and (for DUNE) larger detection cross sections. 

%\begin{deluxetable*}{ccccc}
\begin{deluxetable}{cccccc}
\tablewidth{0pc}
\tablecolumns{6}
\tablecaption{Most Likely Value and Percent Error for Measuring
$L^n_{\nu_e,\mathrm{max}}$, 
  for the 15-\Msol\ Model Employing the \ls, Based on
  95\% Error Bounds, in the No-Oscillation Case \label{tab:maxvalpercenterrortable}}
\tablehead{
\multicolumn{1}{c}{Distance} & \colhead{Super-K} &
\colhead{Hyper-K} & \colhead{DUNE} & \colhead{JUNO}\\
\colhead{(kpc)} & 
\colhead{($10^{58}$ s$^{-1}$)} & \colhead{($10^{58}$ s$^{-1}$)} 
& \colhead{($10^{58}$ s$^{-1}$)} & \colhead{($10^{58}$ s$^{-1}$)}}
\startdata
%1  & 2.0$^{+0.8\%}_{-0.99\%}$ & 2.0$^{+12\%}_{-9.8\%}$ & 2.0$^{+2.3\%}_{-2.3\%}$ & 2.0$^{+2.6\%}_{-2.6\%}$ & 2.0$^{+9.2\%}_{-7.7\%}$\\
%4  & 2.0$^{+2.9\%}_{-2.9\%}$ & 2.0$^{+57\%}_{-31\%}$ & 2.0$^{+9.6\%}_{-7.7\%}$ & 2.0$^{+11\%}_{-8.9\%}$ & 2.0$^{+40\%}_{-27\%}$\\
%7  & 2.0$^{+5.0\%}_{-4.6\%}$ & ... & 2.0$^{+18\%}_{-13\%}$ & 2.0$^{+20\%}_{-15\%}$ & 2.1$^{+101\%}_{-39\%}$\\
%10  & 2.0$^{+7.4\%}_{-6.4\%}$ & ... & 2.0$^{+25\%}_{-18\%}$ & 2.0$^{+29\%}_{-20\%}$ & ...\\
%13.3  & 2.0$^{+10\%}_{-8.4\%}$ & ... & 2.0$^{+33\%}_{-23\%}$ & 2.0$^{+43\%}_{-25\%}$ & ...\\
%16.7  & 2.0$^{+13\%}_{-10\%}$ & ... & 2.0$^{+45\%}_{-26\%}$ & 2.0$^{+64\%}_{-29\%}$ & ...\\
%20  & 2.0$^{+16\%}_{-12\%}$ & ... & 2.0$^{+57\%}_{-31\%}$ & 2.0$^{+98\%}_{-32\%}$ & ...\\
%23.3  & 2.0$^{+19\%}_{-14\%}$ & ... & 2.0$^{+79\%}_{-32\%}$ & ... & ...\\
%26.7  & 2.0$^{+22\%}_{-16\%}$ & ... & 2.0$^{+109\%}_{-36\%}$ & ... & ...\\
%30  & 2.0$^{+24\%}_{-18\%}$ & ... & ... & ... & ...
1  & 2.0$^{+2.3\%}_{-2.3\%}$ & 2.0$^{+2.6\%}_{-2.6\%}$ & 2.0$^{+9.2\%}_{-7.7\%}$\\
4  & 2.0$^{+57\%}_{-31\%}$ & 2.0$^{+9.6\%}_{-7.7\%}$ & 2.0$^{+11\%}_{-8.9\%}$ & 2.0$^{+40\%}_{-27\%}$\\
7  & ... & 2.0$^{+18\%}_{-13\%}$ & 2.0$^{+20\%}_{-15\%}$ & 2.1$^{+101\%}_{-39\%}$\\
10  & ... & 2.0$^{+25\%}_{-18\%}$ & 2.0$^{+29\%}_{-20\%}$ & ...\\
13.3  & ... & 2.0$^{+33\%}_{-23\%}$ & 2.0$^{+43\%}_{-25\%}$ & ...\\
16.7  & ... & 2.0$^{+45\%}_{-26\%}$ & 2.0$^{+64\%}_{-29\%}$ & ...\\
20  & ... & 2.0$^{+57\%}_{-31\%}$ & 2.0$^{+98\%}_{-32\%}$ & ...\\
23.3  & ... & 2.0$^{+79\%}_{-32\%}$ & ... & ...\\
26.7  & ... & 2.0$^{+109\%}_{-36\%}$ & ... & ...\\
30  & ... & ... & ... & ...
\enddata
%\end{deluxetable*}
\end{deluxetable}


Table~\ref{tab:numpeakfit_realparm} shows that, 
 in order to use a measurement of \lmax\ to 
differentiate between progenitor masses assuming the \ls, a
measurement accuracy of $\sim$1-2\% is needed.  Specifically, to
differentiate between a 12-\Msol\ and a 15-\Msol\ progenitor, an
accuracy of $\sim$2\% is needed.  For Super-K and JUNO, in the
 no-oscillation case, this level of
accuracy is not obtained for any of the distances examined in our
analysis.  DUNE and Hyper-K, for a supernova at 1 kpc in 
the no-oscillation case, 
approach this accuracy, but do not quite achieve it. 
More likely is the ability to
differentiate between the \ls\ and the \shen\ by measuring \lmax.
Table~\ref{tab:numpeakfit_realparm} shows that, for the
15-\Msol\ model, an accuracy of \abt 10\% is needed to differentiate
between the two equations of state.  Super-K is close to having this
 accuracy at 1~kpc, JUNO does
obtain this accuracy for supernova distances of \abt 1 kpc and smaller
 in the no-oscillation case, and 
DUNE and Hyper-K for distances of \abt 4 kpc and smaller, 
all in the case of no oscillations.  


%For IceCube, Hyper-K, and DUNE, the percentage value of the
%lower bound of the 95\% uncertainty in the no-oscillation case 
%does not change much as a function
%of model, but the upper bound increases modestly both with model progenitor
%mass and substituting the \shen\ for the \ls.  


\subsubsection{Detector Performance for Measuring \tmax}
Figure~\ref{fig:15maxlocfwhmfuncD} shows the  95\% uncertainty in
measuring \tmax\ (the time of the maximum luminosity of the breakout
burst) in the no-oscillation case as a function of 
distance for the detectors in our
analysis.  
Table~\ref{tab:maxlocpercenterrortable} shows, for each representative
detector and as a function of distance, the mode of the PDF obtained
in each case for \tmax, as well as the 
errors associated with 95\% uncertainty values, in the no-oscillation case.
%This holds for all the progenitor models, with
%an uncertainty range that increases slightly with increasing model
%progenitor mass and with the \shen.  
Hyper-K, in the no-oscillation case,  can
determine \tmax\ to within \abt 1 ms of the model value out to a
distance of \abt 7 kpc.  
%This statement holds for all the models,
%except for the 15-\Msol\ \shen\ model, for which Hyper-K can measure \tmax\
%within \abt 1 ms of the model value out to \abt 14 kpc.
Table~\ref{tab:maxlocpercenterrortable} also shows that the value of
\tmax\ most likely to be measured (the mode of the PDF of \tmax\ in our
analysis) is displaced from the model \tmax\  through many of the 
supernova distances under examination.  
However, this offset of the most-likely measured value is
only a fraction of the error expected in a measurement of \tmax\ in
Hyper-K for reasonable supernova distances ($\gtrsim$7 kpc) 
and so is less important.  
DUNE can measure \tmax\ to an 
accuracy of \abt 1 ms out to \abt 7 kpc in the no-oscillation case.  
Again, for a
given distance the measurement has a possibility of being 
slightly less accurate with
increasing model progenitor mass and more accurate for the \shen.  
JUNO and Super-K, in the no-oscillation case,  
cannot make a measurement within \abt 1 ms of the
model value for a supernova at distances greater 
than \abt 2 kpc.  For all distances and
models, in the no-oscillation case, 
Hyper-K will be the most likely to accurately measure \tmax.

We have defined
\tmax\ in such a way that it is not useful for distinguishing between
progenitor models and equations of state, but an accurate measurement
of \tmax\ in multiple detectors could be useful in
triangulating the position of the supernova.  

\begin{deluxetable}{ccccc}
\tablewidth{0pc}
\tablecolumns{6}
\tablecaption{Most Likely Value and Error for Measuring $t_{\mathrm{max}}$ 
 for the 15-\Msol\ Model Employing the \ls, Based on
  the 95\% Error Bounds, in the No-Oscillation Case\label{tab:maxlocpercenterrortable}}
\tablehead{
\multicolumn{1}{c}{Distance} & \colhead{Super-K} &
\colhead{Hyper-K} & \colhead{DUNE} & \colhead{JUNO}  \\
\colhead{(kpc)} & \colhead{(ms)} & \colhead{(ms)} &
\colhead{(ms)} & \colhead{(ms)}}
\startdata
%1  & 0.08$^{+0.05}_{-0.06}$ & 0.0$^{+0.75}_{-0.59}$ & 0.08$^{+0.13}_{-0.14}$ & 0.06$^{+0.16}_{-0.17}$ & 0.0$^{+0.6}_{-0.44}$\\
%4  & 0.08$^{+0.17}_{-0.17}$ & 0.04$^{+2.3}_{-2.5}$ & 0.05$^{+0.56}_{-0.51}$ & 0.02$^{+0.7}_{-0.63}$ & 0.03$^{+1.9}_{-1.9}$\\
%7  & 0.08$^{+0.29}_{-0.29}$ & ... & 0.0$^{+1.0}_{-0.85}$ & -0.02$^{+1.2}_{-1.1}$ & 0.05$^{+3.2}_{-3.9}$\\
%10  & 0.08$^{+0.42}_{-0.41}$ & ... & 0.0$^{+1.3}_{-1.2}$ & 0.0$^{+1.5}_{-1.6}$ & ...\\
%13.3  & 0.0$^{+0.65}_{-0.47}$ & ... & -0.06$^{+1.7}_{-1.6}$ & 0.0$^{+1.9}_{-2.2}$ & ...\\
%16.7  & 0.02$^{+0.76}_{-0.62}$ & ... & -0.06$^{+2.0}_{-2.0}$ & 0.0$^{+2.3}_{-2.8}$ & ...\\
%20  & 0.02$^{+0.9}_{-0.76}$ & ... & -0.06$^{+2.4}_{-2.4}$ & ... & ...\\
%23.3  & -0.01$^{+1.0}_{-0.84}$ & ... & 0.02$^{+2.7}_{-2.9}$ & ... & ...\\
%26.7  & 0.0$^{+1.2}_{-0.99}$ & ... & 0.0$^{+3.1}_{-4.0}$ & ... & ...\\
%30  & 0.0$^{+1.3}_{-1.1}$ & ... & ... & ... & ...
1  & 0.0$^{+0.75}_{-0.59}$ & 0.08$^{+0.13}_{-0.14}$ & 0.06$^{+0.16}_{-0.17}$ & 0.0$^{+0.6}_{-0.44}$\\
4  & 0.04$^{+2.3}_{-2.5}$ & 0.05$^{+0.56}_{-0.51}$ & 0.02$^{+0.7}_{-0.63}$ & 0.03$^{+1.9}_{-1.9}$\\
7  & ... & 0.0$^{+1.0}_{-0.85}$ & -0.02$^{+1.2}_{-1.1}$ & 0.05$^{+3.2}_{-3.9}$\\
10  & ... & 0.0$^{+1.3}_{-1.2}$ & 0.0$^{+1.5}_{-1.6}$ & ...\\
13.3  & ... & -0.06$^{+1.7}_{-1.6}$ & 0.0$^{+1.9}_{-2.2}$ & ...\\
16.7  & ... & -0.06$^{+2.0}_{-2.0}$ & 0.0$^{+2.3}_{-2.8}$ & ...\\
20  & ... & -0.06$^{+2.4}_{-2.4}$ & ... & ...\\
23.3  & ... & 0.02$^{+2.7}_{-2.9}$ & ... & ...\\
26.7  & ... & 0.0$^{+3.1}_{-4.0}$ & ... & ...\\
30  & ... & ... & ... & ...
\enddata
\end{deluxetable}

\subsubsection{Detector Performance for Measuring $w$}
Figure~\ref{fig:15maxlocfwhmfuncD} shows the 95\% uncertainty in
measuring $w$ (the width of the breakout burst peak) in the
no-oscillation case as a function of distance for the detectors in our
analysis. 
Table~\ref{tab:fwhmpercenterrortable} shows, for each representative
detector and as a function of distance, the mode of the PDF obtained
in each case for $w$ as well as the 
errors associated with 95\% uncertainty values, in the no-oscillation case.
To use a measurement of $w$ to differentiate between the 12-\Msol\ and
15-\Msol\ models using the \ls,
Table~\ref{tab:numpeakfit_realparm}
 shows that $w$ needs to be measured to an accuracy  of $\sim$0.4 ms.
To differentiate between the \ls\ and the \shen\ for the
15-\Msol\ model an accuracy of $\sim$0.6 ms is needed.  However, for
$w$ there appears to be a degeneracy between progenitor mass and
equation of state.  For instance, the 12-\Msol\ model with \ls\ has a
value of $w$ that is close to that of the 15-\Msol\ model with \shen,
much closer than any other two values for the models under
consideration.  Thus, by itself, 
a measurement of $w$ seems unable to
specify a particular progenitor mass and equation of state, but rather
possible combinations of these two.

Super-K and JUNO are unable to make a determination of $w$ to an accuracy of
0.4 ms (the difference between the 15-\Msol\ and 12-\Msol\ models with
the \ls) for any distances under consideration here, 
in the no-oscillation case.  DUNE is close to
being able to measure this accuracy at 1 kpc, but it takes a
detector such Hyper-K before such an accuracy can be achieved for a
supernova at \abt 1 kpc in the no-oscillation case.
For differentiating
between the 15-\Msol\ model and the 20- or 25-\Msol\ models employing
the \ls, the accuracy needed is \abt 0.9 ms.  In the no-oscillation
case, Hyper-K and
 DUNE for supernovae out to \abt 2 kpc, and JUNO/Super-K would be
unable to obtain such an accuracy at any distance examined in this
work.  We thus conclude that a measurement of
$w$ sufficiently accurate to discriminate between supernova progenitor
models is not likely to happen in the event of a galactic supernova,
in the no-oscillation case, 
since the distances needed to obtain a sufficiently accurate
measurement only encompass a minority fraction of the Galaxy.


\begin{deluxetable}{ccccc}
\tablewidth{0pc}
\tablecolumns{6}
\tablecaption{Most Likely Value and Error for Measuring $w$ for the 15-\Msol\ Model Employing
  the \ls, Based on the 95\% Error Bounds, in the No-Oscillation Case\label{tab:fwhmpercenterrortable}}
\tablehead{
\multicolumn{1}{c}{Distance} & \colhead{Super-K} &
\colhead{Hyper-K} & \colhead{DUNE} & \colhead{JUNO}   \\
\colhead{(kpc)} & \colhead{(ms)} & \colhead{(ms)} &
\colhead{(ms)} & \colhead{(ms)}}
\startdata
%1  & 9.4$^{+0.15}_{-0.13}$ & 9.5$^{+2.0}_{-1.8}$ & 9.5$^{+0.42}_{-0.38}$ & 9.5$^{+0.49}_{-0.46}$ & 9.5$^{+1.5}_{-1.4}$\\
%4  & 9.4$^{+0.53}_{-0.47}$ & 8.0$^{+11}_{-3.6}$ & 9.5$^{+1.6}_{-1.5}$ & 9.5$^{+2.0}_{-1.8}$ & 8.3$^{+8.5}_{-3.0}$\\
%7  & 9.5$^{+0.85}_{-0.79}$ & ... & 9.3$^{+3.1}_{-2.3}$ & 9.2$^{+4.0}_{-2.6}$ & ...\\
%10  & 9.5$^{+1.2}_{-1.1}$ & ... & 9.0$^{+4.9}_{-2.7}$ & 8.8$^{+6.4}_{-2.8}$ & ...\\
%13.3  & 9.4$^{+1.7}_{-1.5}$ & ... & 8.3$^{+7.3}_{-2.6}$ & 8.2$^{+9.0}_{-3.1}$ & ...\\
%16.7  & 9.4$^{+2.1}_{-1.9}$ & ... & 8.0$^{+9.4}_{-2.9}$ & 8.0$^{+11}_{-3.8}$ & ...\\
%20  & 9.4$^{+2.7}_{-2.2}$ & ... & ... & ... & ...\\
%23.3  & 9.3$^{+3.2}_{-2.5}$ & ... & ... & ... & ...\\
%26.7  & 9.2$^{+3.8}_{-2.6}$ & ... & ... & ... & ...\\
%30  & 9.1$^{+4.5}_{-2.7}$ & ... & ... & ... & ...
1  & 9.5$^{+2.0}_{-1.8}$ & 9.5$^{+0.42}_{-0.38}$ & 9.5$^{+0.49}_{-0.46}$ & 9.5$^{+1.5}_{-1.4}$\\
4  & 8.0$^{+11}_{-3.6}$ & 9.5$^{+1.6}_{-1.5}$ & 9.5$^{+2.0}_{-1.8}$ & 8.3$^{+8.5}_{-3.0}$\\
7  & ... & 9.3$^{+3.1}_{-2.3}$ & 9.2$^{+4.0}_{-2.6}$ & ...\\
10  & ... & 9.0$^{+4.9}_{-2.7}$ & 8.8$^{+6.4}_{-2.8}$ & ...\\
13.3  & ... & 8.3$^{+7.3}_{-2.6}$ & 8.2$^{+9.0}_{-3.1}$ & ...\\
16.7  & ... & 8.0$^{+9.4}_{-2.9}$ & 8.0$^{+11}_{-3.8}$ & ...\\
20  & ... & ... & ... & ...\\
23.3  & ... & ... & ... & ...\\
26.7  & ... & ... & ... & ...\\
30  & ... & ... & ... & ...
\enddata
\end{deluxetable}


\subsubsection{Detector Performance for Measuring \trise\ and \tfall}
Figure~\ref{fig:15lwhmrwhmfuncD} shows, in the no-oscillation case,
the  95\% uncertainty in
measuring \trise\ (the rise time of the breakout burst luminosity) as
a function of distance for the detectors in our analysis. 
Table~\ref{tab:lwhmpercenterrortable} shows, for each representative
detector and as a function of distance, the mode of the PDF obtained
in each case for \trise\ as well as the 
errors associated with 95\% uncertainty values, in the no-oscillation case.
The values for \trise, for the different progenitor masses employing the
\ls, are all too close to allow a measurement of \trise\ from
any of the detectors under consideration, for any of the supernova
distances under consideration, to differentiate between progenitor
masses, in the no-oscillation case.  
However, Table~\ref{tab:numpeakfit_realparm} shows that there
is a larger difference (\abt 0.5 ms) in \trise\ between the \ls\ and the \shen\ for
the 15-\Msol\ model.  In the no-oscillation case, 
Super-K and JUNO would not make a measurement with this
accuracy for a supernova at any distances considered here (but 
almost could at \abt 1 kpc). DUNE and Hyper-K would achieve this
accuracy for a supernova at \abt
2-3 kpc.

\begin{deluxetable}{ccccc}
\tablewidth{0pc}
\tablecolumns{6}
\tablecaption{Most Likely Value and Error for Measuring
$t_{\mathrm{rise},1/2}$ for the 15-\Msol\ Model Employing the \ls,
Based on the 95\% Error Bounds, in the No-Oscillation Case\label{tab:lwhmpercenterrortable}}
\tablehead{
\multicolumn{1}{c}{Distance} & \colhead{Super-K} &
\colhead{Hyper-K} & \colhead{DUNE} & \colhead{JUNO}  \\
\colhead{(kpc)} & \colhead{(ms)} & \colhead{(ms)} &
\colhead{(ms)} & \colhead{(ms)}}
\startdata
%1  & 2.4$^{+0.08}_{-0.07}$ & 2.4$^{+0.83}_{-0.7}$ & 2.4$^{+0.14}_{-0.17}$ & 2.4$^{+0.19}_{-0.19}$ & 2.4$^{+0.61}_{-0.6}$\\
%4  & 2.4$^{+0.18}_{-0.2}$ & 1.8$^{+3.3}_{-1.0}$ & 2.4$^{+0.67}_{-0.61}$ & 2.4$^{+0.88}_{-0.7}$ & 1.8$^{+2.8}_{-0.81}$\\
%7  & 2.4$^{+0.32}_{-0.3}$ & 1.9$^{+4.1}_{-1.6}$ & 2.3$^{+1.3}_{-0.7}$ & 2.0$^{+1.8}_{-0.55}$ & 1.8$^{+3.8}_{-1.3}$\\
%10  & 2.4$^{+0.46}_{-0.44}$ & ... & 1.9$^{+2.1}_{-0.5}$ & 1.9$^{+2.4}_{-0.65}$ & ...\\
%13.3  & 2.4$^{+0.65}_{-0.63}$ & ... & 1.8$^{+2.6}_{-0.64}$ & 1.8$^{+3.0}_{-0.91}$ & ...\\
%16.7  & 2.4$^{+0.84}_{-0.7}$ & ... & 1.8$^{+3.0}_{-0.83}$ & 1.8$^{+3.3}_{-1.1}$ & ...\\
%20  & 2.4$^{+1.0}_{-0.76}$ & ... & 1.8$^{+3.4}_{-0.99}$ & 1.8$^{+3.5}_{-1.3}$ & ...\\
%23.3  & 2.3$^{+1.2}_{-0.76}$ & ... & 1.7$^{+3.7}_{-1.0}$ & 1.9$^{+3.8}_{-1.4}$ & ...\\
%26.7  & 2.3$^{+1.4}_{-0.78}$ & ... & 1.8$^{+3.7}_{-1.3}$ &
%1.9$^{+3.8}_{-1.6}$ & ...\\
%30  & 2.1$^{+1.7}_{-0.67}$ & ... & 1.8$^{+4.0}_{-1.4}$ &
%1.9$^{+3.9}_{-1.7}$ & ...
1  & 2.4$^{+0.83}_{-0.7}$ & 2.4$^{+0.14}_{-0.17}$ & 2.4$^{+0.19}_{-0.19}$ & 2.4$^{+0.61}_{-0.6}$\\
4  & 1.8$^{+3.3}_{-1.0}$ & 2.4$^{+0.67}_{-0.61}$ & 2.4$^{+0.88}_{-0.7}$ & 1.8$^{+2.8}_{-0.81}$\\
7  & 1.9$^{+4.1}_{-1.6}$ & 2.3$^{+1.3}_{-0.7}$ & 2.0$^{+1.8}_{-0.55}$ & 1.8$^{+3.8}_{-1.3}$\\
10  & ... & 1.9$^{+2.1}_{-0.5}$ & 1.9$^{+2.4}_{-0.65}$ & ...\\
13.3  & ... & 1.8$^{+2.6}_{-0.64}$ & 1.8$^{+3.0}_{-0.91}$ & ...\\
16.7  & ... & 1.8$^{+3.0}_{-0.83}$ & 1.8$^{+3.3}_{-1.1}$ & ...\\
20  & ... & 1.8$^{+3.4}_{-0.99}$ & 1.8$^{+3.5}_{-1.3}$ & ...\\
23.3  & ... & 1.7$^{+3.7}_{-1.0}$ & 1.9$^{+3.8}_{-1.4}$ & ...\\
26.7  & ... & 1.8$^{+3.7}_{-1.3}$ & 1.9$^{+3.8}_{-1.6}$ & ...\\
30  & ... & 1.8$^{+4.0}_{-1.4}$ & 1.9$^{+3.9}_{-1.7}$ & ...
\enddata
\end{deluxetable}


Figure~\ref{fig:15lwhmrwhmfuncD} 
shows, in the no-oscillation case, the 95\% uncertainty in
measuring \tfall\ (the decay time of the breakout burst luminosity) 
as a function of distance for the detectors in our analysis. 
Table~\ref{tab:rwhmpercenterrortable} shows, for each representative
detector and as a function of distance, the mode of the PDF obtained
in each case for \tfall\ as well as the 
errors associated with 95\% uncertainty values, in the no-oscillation case.
The separation of the values for \tfall\ for the \ls\ and the \shen\
for the 15-\Msol\ 
model is too small for any detector or any distance
considered here to have sufficient discriminating power between these
two models, in the no-oscillation case.  
The difference between (for the \ls) the 12- and
15-\Msol\ models is \abt 0.4 ms, and the difference between (for the
\ls) the 20- and 15-\Msol\ models is \abt 0.9 ms. In the
no-oscillation case, DUNE and Hyper-K will be able to
measure \tfall\ with an accuracy of 0.4 ms for distances up to \abt 1
kpc.  JUNO and
Super-K do not achieve this accuracy for any distances in our study.

Measurements of \trise\ and \tfall\ could be used
to show that \tfall$>$\trise.  Table~\ref{tab:numpeakfit_realparm}
shows that \tfall\ is $\sim$3-4 times larger than \trise\ across all
the models.  A measurement of \tfall$>$\trise\ would be important in
verifying current models of the \nue\ breakout burst.   In the
no-oscillation case, Super-K would be able to confirm this out to \abt
2 kpc, JUNO 
would be able to out to \abt 3 kpc, 
DUNE would be able to out to \abt 10 kpc,  and Hyper-K would be able to
out to \abt 11-12 kpc.


\begin{deluxetable}{ccccc}
\tablewidth{0pc}
\tablecolumns{6}
\tablecaption{Most Likely Value and Error for Measuring $t_{\mathrm{fall},1/2}$ for the 15-\Msol\ Model Employing the \ls, Based on the 95\% Error Bounds, in the No-Oscillation Case\label{tab:rwhmpercenterrortable}}
\tablehead{
\multicolumn{1}{c}{Distance} & \colhead{Super-K} &
\colhead{Hyper-K} & \colhead{DUNE} & \colhead{JUNO}  \\
\colhead{(kpc)} & \colhead{(ms)} & \colhead{(ms)} &
\colhead{(ms)} & \colhead{(ms)}}
\startdata
%1  & 7.0$^{+0.11}_{-0.11}$ & 7.0$^{+1.7}_{-1.3}$ & 7.0$^{+0.36}_{-0.29}$ & 7.1$^{+0.41}_{-0.37}$ & 7.0$^{+1.3}_{-1.0}$\\
%4  & 7.0$^{+0.44}_{-0.37}$ & 6.1$^{+8.6}_{-2.7}$ & 7.0$^{+1.4}_{-1.0}$ & 7.0$^{+1.7}_{-1.3}$ & 6.3$^{+6.6}_{-2.3}$\\
%7  & 7.0$^{+0.72}_{-0.63}$ & ... & 6.9$^{+2.6}_{-1.7}$ & 6.8$^{+3.4}_{-1.8}$ & ...\\
%10  & 7.0$^{+1.1}_{-0.86}$ & ... & 6.5$^{+4.3}_{-1.8}$ & 6.4$^{+5.2}_{-2.0}$ & ...\\
%13.3  & 7.0$^{+1.5}_{-1.1}$ & ... & 6.2$^{+5.8}_{-2.0}$ & 6.0$^{+7.1}_{-2.2}$ & ...\\
%16.7  & 6.9$^{+1.9}_{-1.3}$ & ... & 5.9$^{+7.4}_{-2.1}$ & 5.8$^{+8.6}_{-2.6}$ & ...\\
%20  & 6.9$^{+2.3}_{-1.5}$ & ... & 5.6$^{+9.1}_{-2.2}$ & ... & ...\\
%23.3  & 6.8$^{+2.8}_{-1.7}$ & ... & ... & ... & ...\\
%26.7  & 6.7$^{+3.4}_{-1.8}$ & ... & ... & ... & ...\\
%30  & 6.6$^{+3.9}_{-1.9}$ & ... & ... & ... & ...
1  & 7.0$^{+1.7}_{-1.3}$ & 7.0$^{+0.36}_{-0.29}$ & 7.1$^{+0.41}_{-0.37}$ & 7.0$^{+1.3}_{-1.0}$\\
4  & 6.1$^{+8.6}_{-2.7}$ & 7.0$^{+1.4}_{-1.0}$ & 7.0$^{+1.7}_{-1.3}$ & 6.3$^{+6.6}_{-2.3}$\\
7  & ... & 6.9$^{+2.6}_{-1.7}$ & 6.8$^{+3.4}_{-1.8}$ & ...\\
10  & ... & 6.5$^{+4.3}_{-1.8}$ & 6.4$^{+5.2}_{-2.0}$ & ...\\
13.3  & ... & 6.2$^{+5.8}_{-2.0}$ & 6.0$^{+7.1}_{-2.2}$ & ...\\
16.7  & ... & 5.9$^{+7.4}_{-2.1}$ & 5.8$^{+8.6}_{-2.6}$ & ...\\
20  & ... & 5.6$^{+9.1}_{-2.2}$ & ... & ...\\
23.3  & ... & ... & ... & ...\\
26.7  & ... & ... & ... & ...\\
30  & ... & ... & ... & ...
\enddata
\end{deluxetable}


\subsection{Results from Normal Hierarchy Neutrino Oscillations}
In the NH case, the \nue\ flux exchanges with \nuxpart.  This
means the original \nue's no longer dominate in the
electron-scattering cross section, nor do they undergo CC
interactions with \ar40\ nuclei, which are the dominant detection
channels in the detectors under consideration here.  Because of this,
a clear detection of the \nue\ breakout burst is more difficult
in the NH case than in the no-oscillation case.

%For Gd-doped water-\cer\ and scintillation detectors, IBD interactions
%can be tagged.  Additionally, NC scatterings off of oxygen or
%carbon nuclei can be tagged as well, because of the emission of
%photons from the deexcitation of the nucleus.  
%For these proton-rich detectors, subtracting the 
%IBD interactions in the NH case will subtract much of the \background\ without
%subtracting any 
%\nue\ signal.  Subtracting NC scatterings off of oxygen or carbon
%nuclei will subtract some \nue's from the signal but, since these
%interactions have relatively large thresholds (\abt 15-20 MeV), it is
%the \nux's with higher average energy than \nue's 
%that are primarily subtracted and thus
%such a subtraction improves the detectability of the \nue\ breakout signal
%overall.  This helps beat down the \backgrounds.

For Gd-doped water-\cer\ and scintillation detectors, IBD interactions
    can be tagged with high efficiency.  Scintillation detectors can
    tag IBD's with \abt 99\% 
    efficiency, while water-Cherenkov detectors can tag IBD's with
    \abt 90\% efficiency.  The 
    remaining, untagged IBD's can be statistically subtracted using the measured rate of 
    the tagged IBD's.  Additionally, NC scatterings off of oxygen or carbon nuclei can be 
    tagged as well, because of the emission of photons from the de-excitation of the nucleus.  
    For these proton-rich detectors, subtracting the IBD interactions in the NH case will 
    subtract much of the \background\ without subtracting any \nue\ 
    signal.  Subtracting NC scatterings off of oxygen or carbon nuclei will subtract some 
    \nue's from the signal but, since these interactions have relatively large thresholds  
    (~15-20 MeV), it is the \nux's with higher average energy than \nue's that are 
    primarily subtracted and thus such a subtraction improves the detectability of the \nue\ 
    breakout signal overall.  This helps beat down the \backgrounds.  In 
    the case of the \abt 90\% IBD tagging efficiency for water-Cherenkov detectors, the 
    statistical subtraction of the \abt 10\% of IBD's that are not tagged would introduce 
    additional statistical errors.  However, these errors are modest relative to the signal 
    extracted and so for simplicity we assume the ability to tag all of the IBD events, as well 
    as all of the NC scatterings off of oxygen and carbon.
Figure~\ref{fig:hyperk_superk_nh_backgrounds}
shows, in the NH case, the expected count rate in Hyper-K and Super-K 
for a supernova at 4, 7, and
10 kpc and for all neutrinos types, with
neutrinos detected via IBD's and oxygen NC scattering events
subtracted.
Figure~\ref{fig:juno_dune_nh_backgrounds} shows the same for JUNO and
DUNE, except that DUNE has no neutrino signal subtracted and JUNO has
neutrinos detected via IBD's and carbon NC scattering events subtracted.
For Hyper-K in the NH case, the peak of the \nue\ breakout burst 
would not be detectable at 4, 7, or 10 kpc (based on the
size of the error bars relative to the difference in values in each
time bin).  Since 
the fall from the peak is dominated by the 
\backgrounds, fitting the
peak using the procedure in the previous subsection will not
provide an accurate measurement of the properties of the \nue\
breakout burst in the NH case.  The
pre-shock neutronization peak in the NH case is unlikely to be 
discernible due to the expected noise.  

Based on Figures~\ref{fig:hyperk_superk_nh_backgrounds} and \ref{fig:juno_dune_nh_backgrounds}, the peak
will not be discernible in the NH case for either Super-K or JUNO at
any of the distances in that figure (4, 7, and 10 kpc).  The
pre-shock neutronization peak is also unable to be discernible in the
NH due to the expected noise.

%Long-string detectors, unlike scintillation or Gd-doped water-\cer\
%detectors, are
%unable to make any discriminations between neutrinos.  Because of
%this, in the NH case, 
%detection of the \nue\ breakout burst peak is currently impossible for such
%detectors, even for nearby supernovae.


%Figure~\ref{fig:icecube_dune_nh_backgrounds} shows the expected count rate
%in IceCube for all neutrino types and 
%for supernovae at 4, 7, and 10 kpc.  Even for the small noise 
%expected due to the strong
%signal expected from such a large detector and relatively close supernovae, 
%the \nue\ breakout burst peak
%cannot be made out against the \backgrounds.  The pre-shock
%neutronization peak is also unlikely to be seen, based on the noise
%levels 
%relative to the difference in the count rates between adjacent bins.


\ar40\ detectors have as their detection channels CC
absorption of \nue's and \anue's on the \ar40\ nuclei and electron
scattering.  Since, for the NH, all the original \nue\ flux becomes
\nuxpart's, the signal in \ar40\ detectors is dominated by the \nuxpart's that
have become \nue's.  The signal of the original \nue\ flux is lost to
this dominating \nuxpart\ background.  This can be seen in
Figure~\ref{fig:juno_dune_nh_backgrounds}, which shows, in the NH
case, the expected count rate
in DUNE for all neutrino types and 
for supernovae at distances of 4, 7, and 10 kpc.  
The \nue\ breakout burst peak nor the pre-shock
neutronization peak can be made out against the
\backgrounds.  

\subsection{Results from Inverse Hierarchy Neutrino Oscillations}
In the IH hierarchy case, \abt 30\% of the original \nue\ flux remains
intact.  This makes it easier to detect the \nue\ breakout burst
against the \backgrounds\ than in the NH case.  For
Gd-doped water-\cer\ and scintillation detectors (in which signals from
IBD's and oxygen/carbon NC scatterings can be subtracted), a clear peak
should be discernible in an appropriately close supernova (with
``appropriately close'' depending on the size of the detector).  
Figure~\ref{fig:hyperk_superk_ih_backgrounds}
shows, in the IH case, the expected count rate in Hyper-K and Super-K 
for all neutrinos types, with
backgrounds from IBD's and oxygen NC scattering events subtracted, for
supernovae at distances of 4, 7, and 10 kpc.  
Figure~\ref{fig:juno_dune_ih_backgrounds} shows the same for JUNO and
DUNE, except that DUNE has no neutrino signal subtracted and JUNO has
neutrinos detected via IBD's and carbon NC scattering events subtracted.
For all four detectors, a
cleaner peak is seen in the IH case than in the NH case (NH case shown
in Figures~\ref{fig:hyperk_superk_nh_backgrounds} and~\ref{fig:juno_dune_nh_backgrounds}).  For Hyper-K, 
a clear detection of
the \nue\ breakout burst peak in the IH case 
should be possible at 4 kpc, is
marginally possible at 7 kpc, and is unlikely at 10 kpc. The
pre-shock neutronization peak is not likely to be discernible in
Hyper-K at
any of these distances in the IH case.

For Super-K and JUNO,
Figures~\ref{fig:hyperk_superk_ih_backgrounds} and~\ref{fig:juno_dune_ih_backgrounds} show that
the \nue\ peak may be discernible in the IH case for a supernova at 4
kpc, but not likely to be discernible at 7 or 10 kpc.  The pre-shock
neutronization peak is not discernible at any of these
distances in the IH case.

%Despite the partial maintenance of the \nue\ flux relative to the NH
%case, in the IH case 
%long-string detectors (which cannot subtract any \backgrounds) 
%are still unable to see a clearly defined
%peak for the \nue\ breakout burst.  This is seen in  
%Figure~\ref{fig:icecube_dune_ih_backgrounds}, which
%shows the expected count rate in IceCube for all neutrino types in
%the IH case, for
%supernovae at distances of 4, 7, and 10 kpc.
%Even though, in the IH case, \nue's are able to provide a 
%stronger contribution to the
%signal than in the NH case, IBD's due to the \nuxanti's that
%have oscillated to \anue's (Equation~\ref{eq:anue_ih}) provide a
%relatively larger signal that is active through the peak in \nue\
%luminosity, starting \abt 4 ms prior to \tmax.  However, the pre-shock
%neutronization peak may be discernible at 4 kpc in the IH case.

It is the \ar40\ detectors which show the greatest improvement in
measuring the \nue\ signal in the IH case over the NH case.  Since
the cross section for \nue\ absorption on \ar40\ is so large relative to
the other cross sections considered in this work, the partial
maintenance of the original \nue\ flux makes a big difference in
the detectability of the \nue\ signal in these detectors.  
Figure~\ref{fig:juno_dune_ih_backgrounds}
shows, in the IH case, the expected count rate in DUNE 
for all neutrinos types, for
supernovae at distances of 4, 7, and 10 kpc.  In the IH case, 
the \nue\ breakout burst
peak should be discernible at 4 kpc, marginally discernible at 7 kpc,
and not likely discernible 10 kpc.  The pre-breakout neutronization
peak is not discernible  at any of these distances.

Because the IH case allows for certain detectors to have a 
discernible peak, in principle it is also possible for the properties
of the \nue\ breakout burst peak to be measured in the IH case for
those 
supernovae
distances that provide discernible peaks.  We apply the same
analysis outlined in Section~\ref{sec:method} and used in the
no-oscillation case to calculate the accuracy with which the
properties of the breakout burst can be measured by those 
detectors that have the (distance-dependent) ability to measure a
clear peak in luminosity in the IH case.  
These detectors include all the detectors
focused on in this work, minus IceCube.  In doing this, we make no
attempt to correct for the \backgrounds.  We do
take into account the partial oscillation of the \nue\ flux into \nuxpart.
Since the rising \background\ dominate the tail of the peak, we focus
our fitting routine on the peak itself and do not fit the tail past \abt 5
ms after the peak.  A fitting procedure that employs a model to fit
the \backgrounds\ expected in the IH case should provide better accuracy
in measuring the breakout burst peak than the procedure outlined
here.  The rising \background\ have a strong influence on the
luminosity decay from \nue\ peak.  Since we are not accounting for the
\backgrounds\ in our fitting, the value of \tfall\ is significantly
modified by the \backgrounds, more so than \lmax, \tmax,
and \trise.  Because of this, we do not focus on \tfall\ (and $w$,
which depends in part on \tfall) in the IH case.


Figure~\ref{fig:hyperkphysicalparms_hist_ih} shows the PDFs of the
physical parameters derived from the fits to the simulated
observations for Hyper-K in the IH
case. Figure~\ref{fig:superkphysicalparms_hist_ih} shows the same for
Super-K, Figure~\ref{fig:junophysicalparms_hist_ih} shows the same
for JUNO, and Figure~\ref{fig:dunephysicalparms_hist_ih} shows the
same for DUNE, all for the IH case.

Figure~\ref{fig:15funcD_IH} shows the 95\% uncertainty
in measuring \lmax\ (the maximum value of the breakout burst
luminosity) in the IH case, using the analysis outlined above.
Table~\ref{tab:maxvalerrortable_IH}  shows, in the IH case, for each
representative detector (except IceCube) and as a function of
distance, the mode of the PDF obtained in each case for \lmax\ as well
as the percent errors associated with the 95\% uncertainty values.
The uncertainties are larger
at a given distance for a given detector than in the no-oscillation 
case.    This is attributable to the
smaller number of \nue's detected in the IH case, 
relative to the no-oscillation case.  
In general, though, the same hierarchy in the detectors'
ability to measure \lmax\ is seen:  for a given distance, 
Hyper-K (with its larger detection mass) and DUNE (with its larger
detection cross sections)
perform better than the smaller Super-K and
JUNO.
Table~\ref{tab:numpeakfit_realparm} shows that a measurement of \lmax\
needs to have a \abt 1-2\% accuracy to differentiate between the
different models employing the \ls.  This accuracy is not obtained in
the IH case for
any of the detectors in our study for any of the distances we
examine.  However, equations of state may be able to be
differentiated.  Table~\ref{tab:numpeakfit_realparm} shows that, for
the two 15-\Msol\ models, an accuracy of 10\% is needed to
differentiate between the \ls\ and the \shen.  
Super-K and JUNO do not
obtain this accuracy even at 1 kpc for the IH case, 
but DUNE should be able to make
the discrimination at 1 kpc and out to \abt 2 kpc, and
Hyper-K can make this discrimination out to \abt 2-3 kpc.

\begin{deluxetable}{ccccc}
\tablewidth{0pc}
\tablecolumns{5}
\tablecaption{Most Likely Value and Percent Error for Measuring $L^n_{\nu_e,\mathrm{max}}$
  for the 15-\Msol\ Model Employing the \ls, Based on
  95\% Error Bounds, 
for the IH oscillation case\label{tab:maxvalerrortable_IH}}
\tablehead{
\multicolumn{1}{c}{Distance} &  \colhead{Super-K} &
\colhead{Hyper-K} & \colhead{DUNE} & \colhead{JUNO}\\
\colhead{(kpc)}  & 
\colhead{($10^{58}$ s$^{-1}$)} & \colhead{($10^{58}$ s$^{-1}$)} 
& \colhead{($10^{58}$ s$^{-1}$)} & \colhead{($10^{58}$ s$^{-1}$)}}
\startdata
%1  & 2.1$^{+16\%}_{-11\%}$ & 2.0$^{+3.6\%}_{-2.4\%}$ & 2.1$^{+5.5\%}_{-3.6\%}$ & 1.9$^{+17\%}_{-11\%}$\\
%4  & 2.2$^{+95\%}_{-34\%}$ & 2.1$^{+13\%}_{-8.5\%}$ & 2.2$^{+20\%}_{-13\%}$ & ...\\
%7  & ... & 2.1$^{+22\%}_{-14\%}$ & 2.2$^{+39\%}_{-20\%}$ & ...\\
%10  & ... & 2.1$^{+32\%}_{-18\%}$ & 2.3$^{+81\%}_{-26\%}$ & ...\\
%13.3  & ... & 2.2$^{+41\%}_{-24\%}$ & ... & ...\\
%16.7  & ... & 2.2$^{+57\%}_{-30\%}$ & ... & ...\\
%20  & ... & 2.2$^{+96\%}_{-32\%}$ & ... & ...\\
%23.3  & ... & ... & ... & ...\\
%26.7  & ... & ... & ... & ...\\
%30  & ... & ... & ... & ...
1  & 2.1$^{+19\%}_{-14\%}$ & 2.1$^{+3.8\%}_{-3.4\%}$ & 2.1$^{+4.8\%}_{-4.2\%}$ & 1.9$^{+15\%}_{-12\%}$\\
4  & ... & 2.1$^{+16\%}_{-11\%}$ & 2.2$^{+20\%}_{-13\%}$ & ...\\
7  & ... & 2.1$^{+26\%}_{-18\%}$ & 2.2$^{+37\%}_{-21\%}$ & ...\\
10  & ... & 2.2$^{+41\%}_{-23\%}$ & 2.3$^{+79\%}_{-27\%}$ & ...\\
13.3  & ... & 2.2$^{+73\%}_{-28\%}$ & ... & ...\\
16.7  & ... & ... & ... & ...\\
20  & ... & ... & ... & ...\\
23.3  & ... & ... & ... & ...\\
26.7  & ... & ... & ... & ...\\
30  & ... & ... & ... & ...
\enddata
\end{deluxetable}

Figure~\ref{fig:15funcD_IH} shows, in the IH case,  the 95\% uncertainty
in measuring \tmax\ (the time of the maximum luminosity of the 
breakout burst), using the analysis outlined above.  
Table~\ref{tab:maxlocerrortable_IH}  shows, in the IH case, for each
representative detector (except IceCube) and as a function of
distance, the mode of the PDF obtained in each case for \tmax\ as well
as the errors associated with the 95\% uncertainty values.
Similar to \lmax, 
the uncertainties are larger
at a given distance for a given detector than in the no-oscillation 
case.  In particular, 
 the 95\% uncertainties are larger (by a
factor of \abt 2-3) in the IH
oscillation case than in the no-oscillation case.

\begin{deluxetable}{ccccc}
\tablewidth{0pc}
\tablecolumns{5}
\tablecaption{Most Likely Value and Error for Measuring $t_{\mathrm{max}}$
  for the 15-\Msol\ Model Employing the \ls, Based on
  95\% Error Bounds, for the IH oscillation case\label{tab:maxlocerrortable_IH}}
\tablehead{
\multicolumn{1}{c}{Distance} & \colhead{Super-K} &
\colhead{Hyper-K} & \colhead{DUNE} & \colhead{JUNO}  \\
\colhead{(kpc)} & \colhead{(ms)} & \colhead{(ms)} &
\colhead{(ms)} & \colhead{(ms)}}
\startdata
%1  & 0.0$^{+1.3}_{-0.73}$ & 0.04$^{+0.35}_{-0.27}$ & 0.18$^{+0.43}_{-0.48}$ & 0.0$^{+1.4}_{-1.0}$\\
%4  & ... & 0.09$^{+0.95}_{-0.7}$ & 0.2$^{+2.1}_{-1.4}$ & ...\\
%7  & ... & 0.0$^{+2.1}_{-1.0}$ & ... & ...\\
%10  & ... & -0.09$^{+3.8}_{-1.5}$ & ... & ...\\
%13.3  & ... & ... & ... & ...\\
%16.7  & ... & ... & ... & ...\\
%20  & ... & ... & ... & ...\\
%23.3  & ... & ... & ... & ...\\
%26.7  & ... & ... & ... & ...\\
%30  & ... & ... & ... & ...
1  & 0.0$^{+1.9}_{-1.0}$ & -0.05$^{+0.49}_{-0.24}$ & 0.13$^{+0.48}_{-0.43}$ & 0.0$^{+1.4}_{-1.0}$\\
4  & ... & 0.03$^{+1.4}_{-0.86}$ & 0.1$^{+2.2}_{-1.3}$ & ...\\
7  & ... & -0.05$^{+3.2}_{-1.7}$ & ... & ...\\
10  & ... & ... & ... & ...\\
13.3  & ... & ... & ... & ...\\
16.7  & ... & ... & ... & ...\\
20  & ... & ... & ... & ...\\
23.3  & ... & ... & ... & ...\\
26.7  & ... & ... & ... & ...\\
30  & ... & ... & ... & ...
\enddata
\end{deluxetable}

Figure~\ref{fig:15funcD_IH} shows, in the IH case, the  95\% uncertainty
in measuring \trise\ (the rise time of the breakout burst 
luminosity), using the analysis outlined above. 
Table~\ref{tab:lwhmerrortable_IH}  shows, in the IH case, for each
representative detector (except IceCube) and as a function of
distance, the mode of the PDF obtained in each case for \trise\ as well
as the errors associated with the 95\% uncertainty values.
 Similar to \lmax\
and \tmax, \trise\ has larger uncertainties for a given detector at a
given distance in the IH case than in the no-oscillation case.  
Table~\ref{tab:numpeakfit_realparm} shows that using \trise\ to
differentiate between different mass progenitors requires an accuracy
not realized in the IH case by any of the detectors at any of  
the distances examined
here.  For a 15-\Msol\ progenitor, an accuracy of \abt 0.5 ms would be
sufficient to distinguish between the \ls\ and the \shen.  Hyper-K, in
the IH case, can
realize this accuracy at \abt 1 kpc, but not beyond, and none of the other
detectors in the IH case can realize this accuracy at any of the 
distances in this study. 

\begin{deluxetable}{ccccc}
\tablewidth{0pc}
\tablecolumns{5}
\tablecaption{Most Likely Value and Error for Measuring $t_{\mathrm{rise},1/2}$
  for the 15-\Msol\ Model Employing the \ls, Based on
  95\% Error Bounds, for the IH oscillation case\label{tab:lwhmerrortable_IH}}
\tablehead{
\multicolumn{1}{c}{Distance} & \colhead{Super-K} &
\colhead{Hyper-K} & \colhead{DUNE} & \colhead{JUNO}  \\
\colhead{(kpc)} &  \colhead{(ms)} & \colhead{(ms)} &
\colhead{(ms)} & \colhead{(ms)}}
\startdata
%1  & 2.3$^{+1.4}_{-0.77}$ & 2.3$^{+0.51}_{-0.29}$ & 2.3$^{+0.73}_{-0.34}$ & 2.3$^{+1.6}_{-0.84}$\\
%4  & ... & 2.5$^{+0.97}_{-0.82}$ & 2.3$^{+2.8}_{-0.88}$ & ...\\
%7  & ... & 2.2$^{+2.4}_{-0.83}$ & ... & ...\\
%10  & ... & 1.9$^{+4.3}_{-0.81}$ & ... & ...\\
%13.3  & ... & ... & ... & ...\\
%16.7  & ... & ... & ... & ...\\
%20  & ... & ... & ... & ...\\
%23.3  & ... & ... & ... & ...\\
%26.7  & ... & ... & ... & ...\\
%30  & ... & ... & ... & ...
1  & 2.2$^{+2.3}_{-0.81}$ & 2.3$^{+0.65}_{-0.26}$ & 2.7$^{+0.37}_{-0.71}$ & 2.3$^{+1.7}_{-0.81}$\\
4  & ... & 2.4$^{+1.5}_{-0.83}$ & 2.2$^{+2.8}_{-0.84}$ & ...\\
7  & ... & 2.0$^{+3.8}_{-0.93}$ & ... & ...\\
10  & ... & ... & ... & ...\\
13.3  & ... & ... & ... & ...\\
16.7  & ... & ... & ... & ...\\
20  & ... & ... & ... & ...\\
23.3  & ... & ... & ... & ...\\
26.7  & ... & ... & ... & ...\\
30  & ... & ... & ... & ...
\enddata
\end{deluxetable}


% LocalWords:  cccccc 0pc 0pc ccccc 0pc
