In a neutrino detector, for a given neutrino detection channel, the
expected number of detected neutrinos $N_{\textrm{det}}$ as a function of time is given by
\beq
\label{eq:Nint}
N_{\textrm{det}}(t) =  \frac{N_t}{4\pi D^2} \int \frac{dL_\nu(E_\nu,t)}{dE_\nu}\frac{1}{E_\nu}\sigma(E_\nu)\epsilon(E_e)dE_\nu,
\eeq
where $N_t$ is the number of target particles inside the
detector, $dL_\nu/dE_\nu$
is the energy luminosity spectrum of neutrinos, $E_\nu$ is the
neutrino energy, $E_e$ is the energy of the final-state electron, 
$D$ is the distance between the detector and the SN, 
$\sigma$ is the interaction cross section, and $\epsilon$ is
the efficiency of detection.

In a detector, there is not a one-to-one mapping between the energy
of an interacting neutrino and the detected energy of the products.
To understand exactly what
a signal will look like in a detector, we have to understand how the
spectrum of the detectable products relates to the spectrum of the
incident neutrinos.  The number of events produced in a specific
channel with product $x$ with observed kinetic energy $E_x$
is
\begin{equation}
%\frac{dN_y}{dE_y} = \frac{N_t}{4\pi D^2} 
%L_\nu(E_\nu) \frac{d\sigma (E_y\prime,E_\nu)}{dE_y\prime}
\frac{dN_x}{dE_x dE_\nu dt} = \frac{N_t}{4\pi D^2} 
\frac{dL_\nu(t,E_\nu)}{dE_\nu}\frac{1}{E_\nu} \frac{d\sigma
(E_{x}',E_\nu)}{dE_{x}'}, %\delta(t-t'),
\end{equation}
where $E_\nu$ is the energy of the interacting neutrino, $E_x'$ is
the energy of the product (usually an electron or positron),
and $t$ is the detector time (with light-travel time $D/c$ subtracted).  

Neutrino oscillations that occur as a result of the
Mikheyev--Smirnov--Wolfenstein (MSW) effect 
in the envelope
of a star will likely alter the measured \nue\ signal from the
breakout burst.  There are two main realms of flavor conversions to be
considered in the \nue\ breakout burst, connected with the two
potential neutrino-mass hierarchies: the normal hierarchy (NH), 
where the $\nu_3$ mass
eigenstate is of larger mass than either of $\nu_1$ or $\nu_2$; and
the inverted hierarchy (IH), where the $\nu_3$ mass
eigenstate is of smaller mass than either of $\nu_1$ or $\nu_2$. In
the case of the NH, the observed luminosity of a neutrino species
$L^{obs}_{\nu_i}$ is (\citealt{mirizzietal2015})
\begin{align}
L^{\textrm{obs}}_{\nu_e} &= L_{\nu_{\mu,\tau}}, \\
L^{\textrm{obs}}_{\bar \nu_e} &= \cos^2{\theta_{12}}L_{\bar \nu_e}
+ \sin^2{\theta_{12}}L_{\bar \nu_{\mu,\tau}},
\end{align}
where $\theta_{12}$ is the 1, 2 mixing angle.  We use
$\sin^2{\theta_{12}} = 0.308$ in this work (\citealt{oliveetal2014}).
In the case of the IH, the observed luminosity of a neutrino species
is (\citealt{mirizzietal2015})
\begin{align}
L^{\textrm{obs}}_{\nu_e} &= \sin^2{\theta_{12}}L_{\nu_e}
+ \cos^2{\theta_{12}}L_{\nu_{\mu,\tau}}, \\
L^{\textrm{obs}}_{\bar \nu_e} &= L_{\bar \nu_{\mu,\tau}}. \label{eq:anue_ih}
\end{align}


% LocalWords:  det
